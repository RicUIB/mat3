% Options for packages loaded elsewhere
\PassOptionsToPackage{unicode}{hyperref}
\PassOptionsToPackage{hyphens}{url}
%
\documentclass[
]{article}
\usepackage{lmodern}
\usepackage{amssymb,amsmath}
\usepackage{ifxetex,ifluatex}
\ifnum 0\ifxetex 1\fi\ifluatex 1\fi=0 % if pdftex
  \usepackage[T1]{fontenc}
  \usepackage[utf8]{inputenc}
  \usepackage{textcomp} % provide euro and other symbols
\else % if luatex or xetex
  \usepackage{unicode-math}
  \defaultfontfeatures{Scale=MatchLowercase}
  \defaultfontfeatures[\rmfamily]{Ligatures=TeX,Scale=1}
\fi
% Use upquote if available, for straight quotes in verbatim environments
\IfFileExists{upquote.sty}{\usepackage{upquote}}{}
\IfFileExists{microtype.sty}{% use microtype if available
  \usepackage[]{microtype}
  \UseMicrotypeSet[protrusion]{basicmath} % disable protrusion for tt fonts
}{}
\makeatletter
\@ifundefined{KOMAClassName}{% if non-KOMA class
  \IfFileExists{parskip.sty}{%
    \usepackage{parskip}
  }{% else
    \setlength{\parindent}{0pt}
    \setlength{\parskip}{6pt plus 2pt minus 1pt}}
}{% if KOMA class
  \KOMAoptions{parskip=half}}
\makeatother
\usepackage{xcolor}
\IfFileExists{xurl.sty}{\usepackage{xurl}}{} % add URL line breaks if available
\IfFileExists{bookmark.sty}{\usepackage{bookmark}}{\usepackage{hyperref}}
\hypersetup{
  pdftitle={Ejercicios Tema 1 - Probabilidad},
  pdfauthor={Ricardo Alberich, Juan Gabriel Gomila y Arnau Mir},
  hidelinks,
  pdfcreator={LaTeX via pandoc}}
\urlstyle{same} % disable monospaced font for URLs
\usepackage[margin=1in]{geometry}
\usepackage{color}
\usepackage{fancyvrb}
\newcommand{\VerbBar}{|}
\newcommand{\VERB}{\Verb[commandchars=\\\{\}]}
\DefineVerbatimEnvironment{Highlighting}{Verbatim}{commandchars=\\\{\}}
% Add ',fontsize=\small' for more characters per line
\usepackage{framed}
\definecolor{shadecolor}{RGB}{248,248,248}
\newenvironment{Shaded}{\begin{snugshade}}{\end{snugshade}}
\newcommand{\AlertTok}[1]{\textcolor[rgb]{0.94,0.16,0.16}{#1}}
\newcommand{\AnnotationTok}[1]{\textcolor[rgb]{0.56,0.35,0.01}{\textbf{\textit{#1}}}}
\newcommand{\AttributeTok}[1]{\textcolor[rgb]{0.77,0.63,0.00}{#1}}
\newcommand{\BaseNTok}[1]{\textcolor[rgb]{0.00,0.00,0.81}{#1}}
\newcommand{\BuiltInTok}[1]{#1}
\newcommand{\CharTok}[1]{\textcolor[rgb]{0.31,0.60,0.02}{#1}}
\newcommand{\CommentTok}[1]{\textcolor[rgb]{0.56,0.35,0.01}{\textit{#1}}}
\newcommand{\CommentVarTok}[1]{\textcolor[rgb]{0.56,0.35,0.01}{\textbf{\textit{#1}}}}
\newcommand{\ConstantTok}[1]{\textcolor[rgb]{0.00,0.00,0.00}{#1}}
\newcommand{\ControlFlowTok}[1]{\textcolor[rgb]{0.13,0.29,0.53}{\textbf{#1}}}
\newcommand{\DataTypeTok}[1]{\textcolor[rgb]{0.13,0.29,0.53}{#1}}
\newcommand{\DecValTok}[1]{\textcolor[rgb]{0.00,0.00,0.81}{#1}}
\newcommand{\DocumentationTok}[1]{\textcolor[rgb]{0.56,0.35,0.01}{\textbf{\textit{#1}}}}
\newcommand{\ErrorTok}[1]{\textcolor[rgb]{0.64,0.00,0.00}{\textbf{#1}}}
\newcommand{\ExtensionTok}[1]{#1}
\newcommand{\FloatTok}[1]{\textcolor[rgb]{0.00,0.00,0.81}{#1}}
\newcommand{\FunctionTok}[1]{\textcolor[rgb]{0.00,0.00,0.00}{#1}}
\newcommand{\ImportTok}[1]{#1}
\newcommand{\InformationTok}[1]{\textcolor[rgb]{0.56,0.35,0.01}{\textbf{\textit{#1}}}}
\newcommand{\KeywordTok}[1]{\textcolor[rgb]{0.13,0.29,0.53}{\textbf{#1}}}
\newcommand{\NormalTok}[1]{#1}
\newcommand{\OperatorTok}[1]{\textcolor[rgb]{0.81,0.36,0.00}{\textbf{#1}}}
\newcommand{\OtherTok}[1]{\textcolor[rgb]{0.56,0.35,0.01}{#1}}
\newcommand{\PreprocessorTok}[1]{\textcolor[rgb]{0.56,0.35,0.01}{\textit{#1}}}
\newcommand{\RegionMarkerTok}[1]{#1}
\newcommand{\SpecialCharTok}[1]{\textcolor[rgb]{0.00,0.00,0.00}{#1}}
\newcommand{\SpecialStringTok}[1]{\textcolor[rgb]{0.31,0.60,0.02}{#1}}
\newcommand{\StringTok}[1]{\textcolor[rgb]{0.31,0.60,0.02}{#1}}
\newcommand{\VariableTok}[1]{\textcolor[rgb]{0.00,0.00,0.00}{#1}}
\newcommand{\VerbatimStringTok}[1]{\textcolor[rgb]{0.31,0.60,0.02}{#1}}
\newcommand{\WarningTok}[1]{\textcolor[rgb]{0.56,0.35,0.01}{\textbf{\textit{#1}}}}
\usepackage{longtable,booktabs}
% Correct order of tables after \paragraph or \subparagraph
\usepackage{etoolbox}
\makeatletter
\patchcmd\longtable{\par}{\if@noskipsec\mbox{}\fi\par}{}{}
\makeatother
% Allow footnotes in longtable head/foot
\IfFileExists{footnotehyper.sty}{\usepackage{footnotehyper}}{\usepackage{footnote}}
\makesavenoteenv{longtable}
\usepackage{graphicx,grffile}
\makeatletter
\def\maxwidth{\ifdim\Gin@nat@width>\linewidth\linewidth\else\Gin@nat@width\fi}
\def\maxheight{\ifdim\Gin@nat@height>\textheight\textheight\else\Gin@nat@height\fi}
\makeatother
% Scale images if necessary, so that they will not overflow the page
% margins by default, and it is still possible to overwrite the defaults
% using explicit options in \includegraphics[width, height, ...]{}
\setkeys{Gin}{width=\maxwidth,height=\maxheight,keepaspectratio}
% Set default figure placement to htbp
\makeatletter
\def\fps@figure{htbp}
\makeatother
\setlength{\emergencystretch}{3em} % prevent overfull lines
\providecommand{\tightlist}{%
  \setlength{\itemsep}{0pt}\setlength{\parskip}{0pt}}
\setcounter{secnumdepth}{5}

\title{Ejercicios Tema 1 - Probabilidad}
\author{Ricardo Alberich, Juan Gabriel Gomila y Arnau Mir}
\date{Curso de Probabilidad y Variables Aleatorias con R y Python}

\begin{document}
\maketitle

{
\setcounter{tocdepth}{2}
\tableofcontents
}
\hypertarget{ejercicios-de-espacios-muestrales-y-sucesos}{%
\subsection{Ejercicios de Espacios muestrales y
sucesos}\label{ejercicios-de-espacios-muestrales-y-sucesos}}

\begin{enumerate}
\def\labelenumi{\arabic{enumi}.}
\tightlist
\item
  Se seleccionan al azar tres cartas sin reposición de una baraja que
  contiene 3 cartas rojas, 3 azules, 3 verdes y 3 negras. Especifica un
  espacio muestral para este experimento y halla todos los sucesos
  siguientes:

  \begin{itemize}
  \tightlist
  \item
    \(A\) = ``Todas las cartas seleccionadas son rojas''
  \item
    \(B\) = ``Una carta es roja, 1 es verde y otra es azul''
  \item
    \(C\) = ``Salen tres cartas de colores diferentes''
  \end{itemize}
\end{enumerate}

\textbf{Solución} Nuestro espacio muestral será:

\begin{Shaded}
\begin{Highlighting}[]
\KeywordTok{library}\NormalTok{(gtools)}
\KeywordTok{combinations}\NormalTok{(}\DecValTok{4}\NormalTok{, }\DecValTok{3}\NormalTok{, }\KeywordTok{c}\NormalTok{(}\StringTok{'R'}\NormalTok{, }\StringTok{'A'}\NormalTok{, }\StringTok{'V'}\NormalTok{, }\StringTok{'N'}\NormalTok{), }\DataTypeTok{repeats.allowed =} \OtherTok{TRUE}\NormalTok{)}
\end{Highlighting}
\end{Shaded}

\begin{verbatim}
##       [,1] [,2] [,3]
##  [1,] "A"  "A"  "A" 
##  [2,] "A"  "A"  "N" 
##  [3,] "A"  "A"  "R" 
##  [4,] "A"  "A"  "V" 
##  [5,] "A"  "N"  "N" 
##  [6,] "A"  "N"  "R" 
##  [7,] "A"  "N"  "V" 
##  [8,] "A"  "R"  "R" 
##  [9,] "A"  "R"  "V" 
## [10,] "A"  "V"  "V" 
## [11,] "N"  "N"  "N" 
## [12,] "N"  "N"  "R" 
## [13,] "N"  "N"  "V" 
## [14,] "N"  "R"  "R" 
## [15,] "N"  "R"  "V" 
## [16,] "N"  "V"  "V" 
## [17,] "R"  "R"  "R" 
## [18,] "R"  "R"  "V" 
## [19,] "R"  "V"  "V" 
## [20,] "V"  "V"  "V"
\end{verbatim}

\[
A = \left\{RRR\right\}, \quad B = \left\{RVA\right\}, \quad C = \left\{RAV, RAN, RVN, AVN \right\}
\]

\hypertarget{ejercicios-de-probabilidad}{%
\section{Ejercicios de Probabilidad}\label{ejercicios-de-probabilidad}}

\hypertarget{problema-1.}{%
\subsection{Problema 1.}\label{problema-1.}}

Se lanzan al aire dos monedas iguales. Hallar la probabilidad de que
salgan dos caras iguales.

\hypertarget{soluciuxf3n}{%
\subsubsection{Solución}\label{soluciuxf3n}}

Nuestro espacio muestral es:

\begin{Shaded}
\begin{Highlighting}[]
\KeywordTok{permutations}\NormalTok{(}\DecValTok{2}\NormalTok{, }\DecValTok{2}\NormalTok{, }\KeywordTok{c}\NormalTok{(}\StringTok{'c'}\NormalTok{, }\StringTok{'+'}\NormalTok{), }\DataTypeTok{repeats.allowed =} \OtherTok{TRUE}\NormalTok{)}
\end{Highlighting}
\end{Shaded}

\begin{verbatim}
##      [,1] [,2]
## [1,] "+"  "+" 
## [2,] "+"  "c" 
## [3,] "c"  "+" 
## [4,] "c"  "c"
\end{verbatim}

\[
P({cc}) = \frac{\textrm{casos favorables}}{\textrm{casos posibles}}= \frac{\textrm{1}}{4}.
\]

\hypertarget{problema-2.}{%
\subsection{Problema 2.}\label{problema-2.}}

Suponer que se ha trucado un dado de modo que la probabilidad de que
salga un número es proporcional al mismo. a. ) Hallar la probabilidad de
los sucesos elementales, de que salga un número par y también de que
salga un número impar. b. ) Repetir el problema pero suponiendo que la
probabilidad de que salga un determinado número es inversamente
proporcional al mismo.

\hypertarget{soluciuxf3n-1}{%
\subsubsection{Solución}\label{soluciuxf3n-1}}

\textbf{Apartado a)}

El espacio muestral es \(\Omega= \left\{1, 2, 3, 4, 5, 6 \right\}\).

Sabemos que la suma de la probabilidad de todos los sucesos elementales
es \(1\) y que la probabilidad de que salga un número es proporcional al
mismo \(P(k)=\alpha\cdot k\) para \(k=1,2,3,4,5,6\), entonces

\begin{eqnarray*}
1=P(\Omega) & =& P(1) + P(2) + P(3) + P(4) + P(5) + P(6)\\ 
& = & 1\cdot \alpha + 2 \cdot \alpha + 3 \cdot \alpha + 4 \cdot \alpha + 5 \cdot \alpha + 6 \cdot \alpha \\
& =& 21 \cdot \alpha.
\end{eqnarray*}

luego \(\alpha= \frac1{21}.\)

Utilizando este valor las probabilidades de los sucesos elementales son
:

\[
P(1) = \frac{1}{21}, \quad P(2) = \frac{2}{21}, \quad 
P(3) = \frac{3}{21}, \quad 
P(4) = \frac{4}{21}, \quad 
P(5) = \frac{5}{21}, \quad 
P(6) = \frac{6}{21}.
\] La probabilidad de que salga un número par es:

\[
P(\mbox{par}) = P(\{2,4,6\})=P(2) + P(4) + P(6) = \frac{2}{21} + \frac{4}{21} + \frac{6}{21} = \frac{12}{21} = \frac{4}{7}.
\] Y la probabilidad de que salga un número impar:

\[
P(impar) = = P(\{2,4,6\})=P(1) + P(3) + P(5) = \frac{1}{21} + \frac{3}{21} + \frac{5}{21} = \frac{9}{21} = \frac{3}{7}.
\]

\textbf{Apartado b)}

El espacio muestral es el mismo pero ahroa
\(P(k)=\alpha\cdot \frac{1}{k}\) para \(k=1,2,3,4,5,6.\)

\begin{eqnarray*}
1=P(\Omega) &= & P(1) + P(2) + P(3) + P(4) + P(5) + P(6)\\
&=&  \frac{\alpha}{1} + \frac{\alpha}{2} + \frac{\alpha}{3} + \frac{\alpha}{4} + \frac{\alpha}{5} + \frac{\alpha}{6}\\
&=& \frac{60\alpha + 30\alpha + 20\alpha + 15\alpha + 12\alpha + 10\alpha}{60}  \\
&=&\frac{147\alpha}{60},
\end{eqnarray*}

despejando \(\alpha = \frac{60}{147} = \frac{20}{49}\)

La probabilidad de cada suceso elemental es: \[
P(1) = \frac{20}{49},\quad P(2) = \frac{10}{49}, \quad 
P(3) = \frac{20}{147}, \quad 
P(4) = \frac{10}{98}, \quad 
P(5) = \frac{4}{49}, \quad 
P(6) = \frac{3}{98}.
\] Ahora calculamos la probabilidad de par e impar:

\begin{eqnarray*}
P(par) &=& P(\{2,4,6\})=P(2) + P(4) + P(6) = \frac{10}{49} + \frac{10}{98} + \frac{3}{98} = \frac{33}{98}, \\
P(impar) &=&P(\{1,3,5\})= P(1) + P(3) + P(5) = \frac{20}{49} + \frac{20}{147} + \frac{4}{49} = \frac{92}{147}.
\end{eqnarray*}

\hypertarget{problema3}{%
\subsection{Problema3}\label{problema3}}

En una prisión de 100 presos se seleccionan al azar dos personas para
ponerlas en libertad.\\
* a.) ¿Cual es la probabilidad de que el más viejo de los presos sea uno
de los elegidos? * b.) ¿Y que salga elegida la pareja formada por el más
viejo y el más joven?

\hypertarget{soluciuxf3n-2}{%
\subsubsection{Solución}\label{soluciuxf3n-2}}

\textbf{Apartado a)}

Todas las elecciones de dos presos son equiprobables casos posibles
\({100\choose 2}\), mientras que los casos favorables son
\({99\choose 1}\), luego

\[P(\mbox{``que entre 2 presos esté el más viejo"})=\frac{\mbox{CF}}{\mbox{CP}}=
\frac{{99\choose 1}}{{100\choose 2}}=\frac{99}{\frac{ 100\cdot 99}{2}}=\frac{2}{100}=0.02.\]

\begin{Shaded}
\begin{Highlighting}[]
\KeywordTok{choose}\NormalTok{(}\DecValTok{99}\NormalTok{,}\DecValTok{1}\NormalTok{)}\OperatorTok{/}\KeywordTok{choose}\NormalTok{(}\DecValTok{100}\NormalTok{,}\DecValTok{2}\NormalTok{)}
\end{Highlighting}
\end{Shaded}

\begin{verbatim}
## [1] 0.02
\end{verbatim}

\textbf{Apartado b)}

De forma similar ahora solo hay un caso posible

\[P(``\mbox{que entre 2 presos esté el más viejo y el más joven}")=\frac{\mbox{CF}}{\mbox{CP}}=
\frac{1}{{100\choose 2}}=\frac{1}{\frac{100\cdot 99}{2}}=\frac{1}{4950}\approx 0.000202.\]

\begin{Shaded}
\begin{Highlighting}[]
\DecValTok{1}\OperatorTok{/}\KeywordTok{choose}\NormalTok{(}\DecValTok{100}\NormalTok{,}\DecValTok{2}\NormalTok{)}
\end{Highlighting}
\end{Shaded}

\begin{verbatim}
## [1] 0.0002020202
\end{verbatim}

\hypertarget{problema-4}{%
\subsection{Problema 4}\label{problema-4}}

Se apuntan A, B i C a una carrera. ¿Cuál es la probabilidad de que A
acabe antes que C si todos son igual de hábiles corriendo y no puede
haber empates?

\hypertarget{soluciuxf3n-3}{%
\subsubsection{Solución}\label{soluciuxf3n-3}}

Cálculo a fuerza bruta con R:

\begin{Shaded}
\begin{Highlighting}[]
\NormalTok{gtools}\OperatorTok{::}\KeywordTok{permutations}\NormalTok{(}\DataTypeTok{n=}\DecValTok{3}\NormalTok{,}\DataTypeTok{r=}\DecValTok{3}\NormalTok{,}\DataTypeTok{v=}\KeywordTok{c}\NormalTok{(}\StringTok{"A"}\NormalTok{,}\StringTok{"B"}\NormalTok{,}\StringTok{"C"}\NormalTok{))->}\StringTok{ }\NormalTok{maneras}
\NormalTok{maneras}
\end{Highlighting}
\end{Shaded}

\begin{verbatim}
##      [,1] [,2] [,3]
## [1,] "A"  "B"  "C" 
## [2,] "A"  "C"  "B" 
## [3,] "B"  "A"  "C" 
## [4,] "B"  "C"  "A" 
## [5,] "C"  "A"  "B" 
## [6,] "C"  "B"  "A"
\end{verbatim}

\begin{Shaded}
\begin{Highlighting}[]
\NormalTok{contarAC=}\StringTok{ }\ControlFlowTok{function}\NormalTok{(x) }\ControlFlowTok{if}\NormalTok{(}\KeywordTok{which}\NormalTok{(x}\OperatorTok{==}\StringTok{"A"}\NormalTok{)}\OperatorTok{>}\KeywordTok{which}\NormalTok{(x}\OperatorTok{==}\StringTok{"C"}\NormalTok{)) \{ }\KeywordTok{return}\NormalTok{(}\DecValTok{1}\NormalTok{)\} }\ControlFlowTok{else}\NormalTok{ \{}\DecValTok{0}\NormalTok{\}}
\KeywordTok{apply}\NormalTok{(maneras,}\DecValTok{1}\NormalTok{,contarAC)}
\end{Highlighting}
\end{Shaded}

\begin{verbatim}
## [1] 0 0 0 1 1 1
\end{verbatim}

\begin{Shaded}
\begin{Highlighting}[]
\KeywordTok{sum}\NormalTok{(}\KeywordTok{apply}\NormalTok{(maneras,}\DecValTok{1}\NormalTok{,contarAC))}
\end{Highlighting}
\end{Shaded}

\begin{verbatim}
## [1] 3
\end{verbatim}

hay tres maneras de que A acabe antes de C.

\hypertarget{pregunta-5.}{%
\subsection{Pregunta 5.}\label{pregunta-5.}}

En una sala se hallan \(n\) personas. ¿Cual es la probabilidad de que
haya almenos dos personas con el mismo mes de nacimiento? Dar el
resultado para los valores de \(n=3,4,5,6\).

\hypertarget{soluciuxf3n-4}{%
\subsubsection{Solución}\label{soluciuxf3n-4}}

Sea el suceso \(A=\) ``al menos dos han nacido el mismo día'', entonces
\(A^c=\) ``ninguno ha nacido el mismo día''.

Entonces si son \(n\) personas

\$\(P(A^c)=\frac{CF}{CP}=\frac{V_{365}^n}{VR_{365}^n}=\frac{365\cdot 354\cdot \ldots\cdot (365-n+1) }{365^n},\)

haciendo el complementario

\$\(P(A)=1-\frac{365\cdot 354\cdot \ldots\cdot (365-n+1) }{365^n}.\)

\begin{Shaded}
\begin{Highlighting}[]
\NormalTok{prob_mismo_dia=}\ControlFlowTok{function}\NormalTok{(n)\{}\KeywordTok{prod}\NormalTok{((}\DecValTok{365}\OperatorTok{-}\NormalTok{n}\OperatorTok{+}\DecValTok{1}\NormalTok{)}\OperatorTok{:}\DecValTok{365}\NormalTok{)}\OperatorTok{/}\DecValTok{365}\OperatorTok{^}\NormalTok{n\}}
\NormalTok{df=}\KeywordTok{data.frame}\NormalTok{(}\DataTypeTok{n=}\DecValTok{2}\OperatorTok{:}\DecValTok{40}\NormalTok{,}\DataTypeTok{prob=}\KeywordTok{sapply}\NormalTok{(}\DecValTok{2}\OperatorTok{:}\DecValTok{40}\NormalTok{, prob_mismo_dia))}
\NormalTok{knitr}\OperatorTok{::}\KeywordTok{kable}\NormalTok{(df,}\DataTypeTok{digits =} \DecValTok{4}\NormalTok{)}
\end{Highlighting}
\end{Shaded}

\begin{longtable}[]{@{}rr@{}}
\toprule
n & prob\tabularnewline
\midrule
\endhead
2 & 0.9973\tabularnewline
3 & 0.9918\tabularnewline
4 & 0.9836\tabularnewline
5 & 0.9729\tabularnewline
6 & 0.9595\tabularnewline
7 & 0.9438\tabularnewline
8 & 0.9257\tabularnewline
9 & 0.9054\tabularnewline
10 & 0.8831\tabularnewline
11 & 0.8589\tabularnewline
12 & 0.8330\tabularnewline
13 & 0.8056\tabularnewline
14 & 0.7769\tabularnewline
15 & 0.7471\tabularnewline
16 & 0.7164\tabularnewline
17 & 0.6850\tabularnewline
18 & 0.6531\tabularnewline
19 & 0.6209\tabularnewline
20 & 0.5886\tabularnewline
21 & 0.5563\tabularnewline
22 & 0.5243\tabularnewline
23 & 0.4927\tabularnewline
24 & 0.4617\tabularnewline
25 & 0.4313\tabularnewline
26 & 0.4018\tabularnewline
27 & 0.3731\tabularnewline
28 & 0.3455\tabularnewline
29 & 0.3190\tabularnewline
30 & 0.2937\tabularnewline
31 & 0.2695\tabularnewline
32 & 0.2467\tabularnewline
33 & 0.2250\tabularnewline
34 & 0.2047\tabularnewline
35 & 0.1856\tabularnewline
36 & 0.1678\tabularnewline
37 & 0.1513\tabularnewline
38 & 0.1359\tabularnewline
39 & 0.1218\tabularnewline
40 & 0.1088\tabularnewline
\bottomrule
\end{longtable}

\hypertarget{problema-6.}{%
\subsection{Problema 6.}\label{problema-6.}}

Una urna contiene 4 bolas numeradas con los números 1, 2, 3 y 4,
respectivamente. Se sacan dos bolas sin reposición. Sea \(A\) el suceso
que la suma sea 5 y sea \(B_i\) el suceso que la primera bola extraida
tenga un \(i\), con \(i=1,2,3,4\). Hallar \(P(A/B_i)\) y \(P(B_i/A)\),
para \(i=1,2,3,4\).

\hypertarget{soluciuxf3n-5}{%
\subsubsection{Solución}\label{soluciuxf3n-5}}

Los casos equiprobables y el valor de su suma son

\begin{Shaded}
\begin{Highlighting}[]
\NormalTok{resultados=gtools}\OperatorTok{::}\KeywordTok{permutations}\NormalTok{(}\DecValTok{4}\NormalTok{,}\DataTypeTok{r=}\DecValTok{2}\NormalTok{)}
\NormalTok{df_resultados=}\KeywordTok{data.frame}\NormalTok{(}\DataTypeTok{bola1=}\NormalTok{resultados[,}\DecValTok{1}\NormalTok{],}\DataTypeTok{bola2=}\NormalTok{resultados[,}\DecValTok{2}\NormalTok{],}\DataTypeTok{suma=}\KeywordTok{unlist}\NormalTok{(}\KeywordTok{apply}\NormalTok{(resultados,}\DecValTok{1}\NormalTok{,sum)))}
\NormalTok{df_resultados}
\end{Highlighting}
\end{Shaded}

\begin{verbatim}
##    bola1 bola2 suma
## 1      1     2    3
## 2      1     3    4
## 3      1     4    5
## 4      2     1    3
## 5      2     3    5
## 6      2     4    6
## 7      3     1    4
## 8      3     2    5
## 9      3     4    7
## 10     4     1    5
## 11     4     2    6
## 12     4     3    7
\end{verbatim}

\begin{Shaded}
\begin{Highlighting}[]
\KeywordTok{table}\NormalTok{(df_resultados}\OperatorTok{$}\NormalTok{suma)}
\end{Highlighting}
\end{Shaded}

\begin{verbatim}
## 
## 3 4 5 6 7 
## 2 2 4 2 2
\end{verbatim}

Con esta tabla es sencillo calcular las probabilidades pedidas

\begin{eqnarray*}
P(A|B_1) &=& \frac{P(A \cap B_1)}{P(B_1)} = \frac{1/12}{1/4} = \frac{1}{3}\\
P(A|B_2) &=& \frac{P(A \cap B_2)}{P(B_2)} = \frac{1/12}{1/4} = \frac{1}{3}\\
P(A|B_3) &=& \frac{P(A \cap B_3)}{P(B_3)} = \frac{1/12}{1/4} = \frac{1}{3}\\
P(A|B_4) &=& \frac{P(A \cap B_4)}{P(B_4)} = \frac{1/12}{1/4} = \frac{1}{3}\\
P(B_1|A) &=& \frac{P(B_1 \cap A)}{P(A)} = \frac{1/12}{4/12} = \frac{1}{4}\\
P(B_2|A) &=& \frac{P(B_2 \cap A)}{P(A)} = \frac{1/12}{4/12} = \frac{1}{4}\\
P(B_3|A) &=& \frac{P(B_3 \cap A)}{P(A)} = \frac{1/12}{4/12} = \frac{1}{4}\\
P(B_4|A) &=& \frac{P(B_4 \cap A)}{P(A)} = \frac{1/12}{4/12} = \frac{1}{4}\\
\end{eqnarray*}

\hypertarget{problema-7}{%
\subsection{Problema 7}\label{problema-7}}

Se lanza al aire una moneda no trucada.

\begin{itemize}
\item
  \begin{enumerate}
  \def\labelenumi{\alph{enumi})}
  \tightlist
  \item
    ¿Cuál es la probabilidad que la cuarta vez salga cara, si sale cara
    en las tres primeras tiradas?
  \end{enumerate}
\item
  \begin{enumerate}
  \def\labelenumi{\alph{enumi})}
  \setcounter{enumi}{1}
  \tightlist
  \item
    ¿Y si salen 2 caras en las 4 tiradas?
  \end{enumerate}
\end{itemize}

\hypertarget{soluciuxf3n-6}{%
\subsection{Solución}\label{soluciuxf3n-6}}

\textbf{Apartado a)}

Denotemos por \(c\) la cara y por \(+\) la cruz y
\(P(c)=P(+)=\frac{1}{2}\). Y denotemos por cadanas las sucesivas tiradas
\(cccc,ccc+,\ldots\).

Dado que son sucesos independientes, nos da igual cuántas veces hayamos
tirado el dado y lo que haya salido las veces anteriores Luego
\(P("\mbox{4 cuarta tirada cara}"/ccc)=P("\mbox{4 cuarta tirada cara}"/ccc)=\frac{1}{2}.\)

\textbf{Apartado b)}

Nuevamente, al ser sucesos independientes, nuestra probabilidad de cara
será:

\begin{eqnarray*}
P(``\mbox{4 cuarta  tirada  cara}"&/& ``\mbox{si han salido  3 caras en 4 tiradas}") \\ 
& =& P(\{cccc,ccc+,cc+c,c+cc,cc++,c+c+,c++c,c+++\}/\{cc++,c+c+,c++c,+c+c,++cc\})\\
&=& \frac{P(\{cccc,ccc+,cc+c,c+cc,cc++,c+c+,c++c,c+++\}\\&\cap &\{cc++,c+c+,c++c,+c+c,++cc\})}{P(\{cc++,c+c+,c++c,+c+c,++cc\})}\\
&=& \frac{P(\{cc++,c+c+,c++c\})}{P(\{cc++,c+c+,c++c,+c+c,++cc\})}=\frac{\frac{3}{16}}{\frac{5}{16}}=\frac{3}{5}.
\end{eqnarray*}

\hypertarget{problema-8.}{%
\subsection{Problema 8.}\label{problema-8.}}

La urna 1 contiene 2 bolas rojas y 4 de azules. La urna 2 contiene 10
bolas rojas y 2 de azules. Si escogemos al azar una urna y sacamos una
bola,

\begin{itemize}
\item
  \begin{enumerate}
  \def\labelenumi{\alph{enumi})}
  \tightlist
  \item
    ¿Cuál es la probabilidad que la bola seleccionada sea azul?
  \end{enumerate}
\item
  \begin{enumerate}
  \def\labelenumi{\alph{enumi})}
  \setcounter{enumi}{1}
  \tightlist
  \item
    ¿Y que sea roja?
  \end{enumerate}
\end{itemize}

\textbf{Solución}

Definimos los sucesos:

\begin{itemize}
\tightlist
\item
  U1: Escoger la urna 1
\item
  U2: Escoger la urna 2
\item
  R: Escoger una bola roja
\item
  A: Escoger una bola azul
\end{itemize}

\textbf{Apartado a)}

\[
P(A) = P(U1) \cdot P(A|U1) + P(2) \cdot P(A|U2) = \frac{1}{2}\cdot \frac{4}{6} + \frac{1}{2} \cdot \frac{2}{12} = 0.4166
\]

\textbf{Apartado b)}

\[
P(R) = 1 - P(R^c) = 1-P(A)= 1 - 0.41666 = 0.5833
\]

\hypertarget{problema-8}{%
\subsection{Problema 8}\label{problema-8}}

Supongamos que la ciencia médica ha desarrollado una prueba para el
diagnóstico de cáncer que tiene un 95\% de exactitud, tanto en los que
tienen cáncer como en los que no. Si el 5 por mil de la población
realmente tiene cáncer, encontrar la probabilidad que un determinado
individuo tenga cáncer, si la prueba ha dado positiva.

\textbf{Solución}

Definimos los sucesos:

\begin{itemize}
\tightlist
\item
  +: Prueba da positivo
\item
  C: Tener cancer
\end{itemize}

\[
\begin{eqnarray*}
P(C|+) &=& \frac{P(C) \cdot P(+|C)}{P(+)} =
\frac{P(C) \cdot P(+|C)}{P(C) \dot P(+|C) + P(C^c) \cdot P(+|C^c)} \\
&=&  \frac{\frac5{1000} \cdot 0.95}{\frac5{1000} \cdot 0.95 + \frac{995}{1000} \cdot 0.05} = 0.087115596.
\end{eqnarray*}
\]

\hypertarget{problema-9.}{%
\subsection{Problema 9.}\label{problema-9.}}

Se lanzan una sola vez dos dados. Si la suma de los dos dados es como
mínimo 7, ¿cuál es la probabilidad que la suma sea igual a \(i\), para
\(i=7,8,9,10,11,12\)?

\hypertarget{soluciuxf3n-7}{%
\subsubsection{Solución}\label{soluciuxf3n-7}}

Los casos son

\begin{Shaded}
\begin{Highlighting}[]
\NormalTok{dados=gtools}\OperatorTok{::}\KeywordTok{permutations}\NormalTok{(}\DataTypeTok{n=}\DecValTok{6}\NormalTok{,}\DataTypeTok{r=}\DecValTok{2}\NormalTok{,}\DataTypeTok{repeats.allowed =} \OtherTok{TRUE}\NormalTok{)}
\NormalTok{df_dados=}\KeywordTok{data.frame}\NormalTok{(}\DataTypeTok{dado1=}\NormalTok{dados[,}\DecValTok{1}\NormalTok{],}\DataTypeTok{dado2=}\NormalTok{dados[,}\DecValTok{2}\NormalTok{],}\DataTypeTok{suma=}\NormalTok{dados[,}\DecValTok{1}\NormalTok{]}\OperatorTok{+}\NormalTok{dados[,}\DecValTok{2}\NormalTok{])}
\NormalTok{df_dados}
\end{Highlighting}
\end{Shaded}

\begin{verbatim}
##    dado1 dado2 suma
## 1      1     1    2
## 2      1     2    3
## 3      1     3    4
## 4      1     4    5
## 5      1     5    6
## 6      1     6    7
## 7      2     1    3
## 8      2     2    4
## 9      2     3    5
## 10     2     4    6
## 11     2     5    7
## 12     2     6    8
## 13     3     1    4
## 14     3     2    5
## 15     3     3    6
## 16     3     4    7
## 17     3     5    8
## 18     3     6    9
## 19     4     1    5
## 20     4     2    6
## 21     4     3    7
## 22     4     4    8
## 23     4     5    9
## 24     4     6   10
## 25     5     1    6
## 26     5     2    7
## 27     5     3    8
## 28     5     4    9
## 29     5     5   10
## 30     5     6   11
## 31     6     1    7
## 32     6     2    8
## 33     6     3    9
## 34     6     4   10
## 35     6     5   11
## 36     6     6   12
\end{verbatim}

\begin{Shaded}
\begin{Highlighting}[]
\KeywordTok{table}\NormalTok{(df_dados}\OperatorTok{$}\NormalTok{suma)}
\end{Highlighting}
\end{Shaded}

\begin{verbatim}
## 
##  2  3  4  5  6  7  8  9 10 11 12 
##  1  2  3  4  5  6  5  4  3  2  1
\end{verbatim}

A partir de esta tabla es sencillo cacular las siguientes
probabilidades:

\[
P(suma = 7|suma \geq 7) = \frac{P(suma =7 \cap suma \geq 7)}{P(suma \geq 7)} = \\
\frac{6/36}{6/36 + 5/36 + 4/36 + 3/36 + 2/36 + 1/36} = \frac{6}{21}
\] \[
P(suma = 8|suma \geq 7) = \frac{P(suma = 8 \cap suma \geq 7)}{P(suma \geq 7)} = \\
\frac{5/36}{6/36 + 5/36 + 4/36 + 3/36 + 2/36 + 1/36} = \frac{5}{21}
\] \[
P(suma = 9|suma \geq 7) = \frac{P(suma =9 \cap suma \geq 7)}{P(suma \geq 7)} = \\
\frac{4/36}{6/36 + 5/36 + 4/36 + 3/36 + 2/36 + 1/36} = \frac{4}{21}
\] \[
P(suma = 10|suma \geq 7) = \frac{P(suma = 10 \cap suma \geq 7)}{P(suma \geq 7)} = \\
\frac{3/36}{6/36 + 5/36 + 4/36 + 3/36 + 2/36 + 1/36} = \frac{3}{21}
\]

\[
P(suma = 11|suma \geq 7) = \frac{P(suma = 11 \cap suma \geq 7)}{P(suma \geq 7)} = \\
\frac{2/36}{6/36 + 5/36 + 4/36 + 3/36 + 2/36 + 1/36} = \frac{2}{21}
\] \[
P(suma = 12|suma \geq 7) = \frac{P(suma =12 \cap suma \geq 7)}{P(suma \geq 7)} = \\
\frac{1/36}{6/36 + 5/36 + 4/36 + 3/36 + 2/36 + 1/36} = \frac{1}{21}
\]

\hypertarget{problema-10.}{%
\subsection{Problema 10.}\label{problema-10.}}

Se sabe que \({2 \over 3}\) de los internos de una cierta prisión son
menores de 25 años. También se sabe que \({3\over 5}\) son hombres y que
\({5\over 8}\) de los internos son mujeres o mayores de 25 años. ¿Cuál
es la probabilidad de que un prisionero escogido al azar sea mujer y
menor de 25 años?

\hypertarget{soluciuxf3n-8}{%
\subsubsection{Solución}\label{soluciuxf3n-8}}

Sean los sucesos:

\begin{itemize}
\tightlist
\item
  M: interno menor de 25 años
\item
  H: interno hombre
\end{itemize}

Nos dan los siguientes datos:

\begin{itemize}
\tightlist
\item
  \(P(M) = \frac{2}{3}\)
\item
  \(P(H) = \frac{3}{5}\)
\item
  \(P(M^c \cup H^c) = \frac{5}{8}\)
\end{itemize}

Queremos calcular \(P(M\cap H^c)\):

\[
P(M\cap H^c) = P(M) - P(M \cap H) = \\
P(M) - (1 - P((M \cap H)^c)) = P(M) - (1 - P(M^c \cup H^c)) = \\
\frac{2}{3} - \left(1 - \frac{5}{8} \right) = 0.29
\]

\hypertarget{problema-11}{%
\subsection{Problema 11}\label{problema-11}}

Consideremos una hucha con \(2n\) bolas numeradas del \(1\) al \(2n\).
Sacamos \(2\) bolas de la urna sin reposición. Sabiendo que la segunda
bola es par, ¿cuál es la probabilidad de que la primera bola sea impar?

\textbf{Solución}

Definimos los sucesos:

\begin{itemize}
\tightlist
\item
  \(P_2\): que la segunda bola sea par
\item
  \(I_1\): que la primera bola se impar
\end{itemize}

\[
P(I_1 | P_2) = \frac{P(P_2 \cap I_1)}{P(P_2)}
\]

Como el número de bolas siempre es par porque son \(2n\) sabemos que
siempre va a haber el mismo número de bolas pares que impares, por tanto
la probabilidad de que la segunda bola sea par será:

\[
P(P_2) = \frac{1}{2}
\] \[
P(P_2 \cap I_1)
\] Dependerá del valor de \(n\). En cualquier caso,

\[
P(P_2 \cap I_1) = \frac{\textrm{casos favorables}}{\textrm{casos posibles}}
\] Los casos posibles serán variaciones 2 bolas de un conjunto de \(2n\)
bolas:

\[
V^{2n}_2 = \frac{(2n)!}{(2n - 2)!} = 2n\cdot (2n-1) 
\]

Los casos posibles serán \(n^2\). Por tanto lo podemos reescribir todo
de la siguiente manera:

\[
P(I_1|P_2) = \frac{P(P_2 \cap I_1)}{P(P_2)} = \frac{\frac{n^2}{\frac{(2n))!}{(2n - 2)!}}}{1/2} = \frac{n}{2n-1}
\]

\hypertarget{problema-12.}{%
\subsubsection{Problema 12.}\label{problema-12.}}

Consideramos el siguiente experimento aleatorio: sacamos \(5\) números
al azar sin reposición a partir de los números naturales
\(1,2,\dots,20\). Encontrad la probabilidad \(p\) de que haya
exactamente dos números tales que sean múltiplos de \(3\)

\hypertarget{soluciuxf3n-9}{%
\subsubsection{Solución}\label{soluciuxf3n-9}}

Múltiplos de 3 \(=\dot 3= \{3, 6, 9, 12, 15, 18\}.\)

\[
P(\mbox{``dos multiplos de 3"} )=\frac{CF}{CP}=\frac{{6\choose2}\cdot{{20-6}\choose {3}}}{{20\choose 5}}=\frac{90}{ 15504}.=0.3521672.
\]

Con R:

\begin{Shaded}
\begin{Highlighting}[]
\KeywordTok{choose}\NormalTok{(}\DecValTok{6}\NormalTok{,}\DecValTok{2}\NormalTok{)}
\end{Highlighting}
\end{Shaded}

\begin{verbatim}
## [1] 15
\end{verbatim}

\begin{Shaded}
\begin{Highlighting}[]
\KeywordTok{choose}\NormalTok{(}\DecValTok{20-6}\NormalTok{,}\DecValTok{3}\NormalTok{)}
\end{Highlighting}
\end{Shaded}

\begin{verbatim}
## [1] 364
\end{verbatim}

\begin{Shaded}
\begin{Highlighting}[]
\KeywordTok{choose}\NormalTok{(}\DecValTok{20}\NormalTok{,}\DecValTok{5}\NormalTok{)}
\end{Highlighting}
\end{Shaded}

\begin{verbatim}
## [1] 15504
\end{verbatim}

\begin{Shaded}
\begin{Highlighting}[]
\KeywordTok{choose}\NormalTok{(}\DecValTok{6}\NormalTok{,}\DecValTok{2}\NormalTok{)}\OperatorTok{*}\StringTok{ }\KeywordTok{choose}\NormalTok{(}\DecValTok{20-6}\NormalTok{,}\DecValTok{3}\NormalTok{)}\OperatorTok{/}\KeywordTok{choose}\NormalTok{(}\DecValTok{20}\NormalTok{,}\DecValTok{5}\NormalTok{)}
\end{Highlighting}
\end{Shaded}

\begin{verbatim}
## [1] 0.3521672
\end{verbatim}

\hypertarget{problema-13}{%
\subsection{Problema 13}\label{problema-13}}

En una hucha hay \(10\) bolas, numeradas del \(1\) al \(10\). Las \(4\)
primeras bolas, o sea, las bolas \(1,2,3,4\) son blancas. Las bolas
\(5,6\) son negras y las bolas restantes son rojas. Sacamos dos bolas
sin reposición. Sabiendo que la segunda bola es de color negro,
encuentra la probabilidad \(p\) de que la primera bola sea blanca.

\textbf{Solución}

Definimos los sucesos:

\begin{itemize}
\tightlist
\item
  \(B_i\): que la bola \(i\) sea blanca, para \(i=1,2.\)
\item
  \(N_2\): que la bola \(i\) sea negra, para \(i=1,2.\)
\end{itemize}

\begin{eqnarray*}
P(B_1|N_2) &=& \frac{P(B_1 \cap N_2)}{P(N_2)} = 
\frac{P(B_1 \cap N_2)}{P(N_1^c)\cdot P(N_2 |N_1^c) + P(N_1) \cdot P(N_2|N_1)} \\
&=& 
\frac{4/10 \cdot 2/9}{8/10 \cdot 8/9 + 2/10 \cdot 1/9} = 0.4444
\end{eqnarray*}

\hypertarget{problema-14}{%
\subsection{Problema 14}\label{problema-14}}

Lanzamos un dado no trucado 3 veces. Encontrad la probabilidad \(p\) de
que la suma de las 3 caras sea \(10\).

\hypertarget{soluciuxf3n-10}{%
\subsubsection{Solución}\label{soluciuxf3n-10}}

Lo calculamos con R

\begin{Shaded}
\begin{Highlighting}[]
\NormalTok{casos_tres_lanzamientos=gtools}\OperatorTok{::}\KeywordTok{permutations}\NormalTok{(}\DataTypeTok{n=}\DecValTok{6}\NormalTok{,}\DataTypeTok{r=}\DecValTok{3}\NormalTok{,}\DataTypeTok{repeats.allowed =} \OtherTok{TRUE}\NormalTok{)}
\NormalTok{df_tres_lanzamientos=}\KeywordTok{data.frame}\NormalTok{(casos_tres_lanzamientos, }\DataTypeTok{suma=}\KeywordTok{unlist}\NormalTok{(}\KeywordTok{apply}\NormalTok{(casos_tres_lanzamientos,}\DecValTok{1}\NormalTok{,sum)))}
\KeywordTok{head}\NormalTok{(df_tres_lanzamientos)}
\end{Highlighting}
\end{Shaded}

\begin{verbatim}
##   X1 X2 X3 suma
## 1  1  1  1    3
## 2  1  1  2    4
## 3  1  1  3    5
## 4  1  1  4    6
## 5  1  1  5    7
## 6  1  1  6    8
\end{verbatim}

\begin{Shaded}
\begin{Highlighting}[]
\KeywordTok{dim}\NormalTok{(df_tres_lanzamientos)}
\end{Highlighting}
\end{Shaded}

\begin{verbatim}
## [1] 216   4
\end{verbatim}

\begin{Shaded}
\begin{Highlighting}[]
\KeywordTok{table}\NormalTok{(df_tres_lanzamientos}\OperatorTok{$}\NormalTok{suma)}
\end{Highlighting}
\end{Shaded}

\begin{verbatim}
## 
##  3  4  5  6  7  8  9 10 11 12 13 14 15 16 17 18 
##  1  3  6 10 15 21 25 27 27 25 21 15 10  6  3  1
\end{verbatim}

\begin{Shaded}
\begin{Highlighting}[]
\NormalTok{casos_favorables=}\DecValTok{27}
\NormalTok{casos_posibles=}\KeywordTok{dim}\NormalTok{(df_tres_lanzamientos)[}\DecValTok{1}\NormalTok{]}
\NormalTok{casos_posibles}
\end{Highlighting}
\end{Shaded}

\begin{verbatim}
## [1] 216
\end{verbatim}

\begin{Shaded}
\begin{Highlighting}[]
\NormalTok{casos_favorables}\OperatorTok{/}\NormalTok{casos_posibles}
\end{Highlighting}
\end{Shaded}

\begin{verbatim}
## [1] 0.125
\end{verbatim}

\hypertarget{problema-15}{%
\subsection{Problema 15}\label{problema-15}}

cartas numeradas del 1 al 4 están giradas boca abajo sobre una mesa. Una
persona, supuestamente adivina, irá adivinando los valores de las 4
cartas una a una. Suponiendo que es un farsante y que lo que hace es
decir los 4 números al azar, ¿cuál es la probabilidad de que acierte
como mínimo 1? (Obviamente, no repite ningún número)

\hypertarget{soluciuxf3n-11}{%
\subsection{Solución}\label{soluciuxf3n-11}}

A fuerza bruta con R:

\begin{Shaded}
\begin{Highlighting}[]
\NormalTok{cartas =}\StringTok{ }\KeywordTok{c}\NormalTok{(}\DecValTok{1}\NormalTok{, }\DecValTok{2}\NormalTok{, }\DecValTok{3}\NormalTok{, }\DecValTok{4}\NormalTok{)}
\NormalTok{permutaciones =}\StringTok{ }\KeywordTok{permutations}\NormalTok{(}\DecValTok{4}\NormalTok{, }\DecValTok{4}\NormalTok{)}
\NormalTok{casos_favorables =}\StringTok{ }\DecValTok{0}
\ControlFlowTok{for}\NormalTok{(ind_caso }\ControlFlowTok{in} \DecValTok{1}\OperatorTok{:}\KeywordTok{nrow}\NormalTok{(permutaciones))\{}
\NormalTok{  caso =}\StringTok{ }\NormalTok{permutaciones[ind_caso, ]}
  \ControlFlowTok{if}\NormalTok{(}\KeywordTok{sum}\NormalTok{(caso }\OperatorTok{==}\StringTok{ }\NormalTok{cartas) }\OperatorTok{>=}\StringTok{ }\DecValTok{1}\NormalTok{)\{}
\NormalTok{    casos_favorables =}\StringTok{ }\NormalTok{casos_favorables }\OperatorTok{+}\StringTok{ }\DecValTok{1}
\NormalTok{  \}}
\NormalTok{\}}

\NormalTok{casos_favorables}\OperatorTok{/}\KeywordTok{nrow}\NormalTok{(permutaciones)}
\end{Highlighting}
\end{Shaded}

\begin{verbatim}
## [1] 0.625
\end{verbatim}

Calculándolo:

\[
P(\textrm{acierte al menos 1} = 1 - P(\textrm{no acierte ninguna})) = \frac{3\cdot 3 \cdot 1 \cdot 1}{4!} = \frac{15}{24}
\] \#\# Problema 16

Una forma de aumentar la fiabilidad de un sistema es mediante la
introducción de una copia de los componentes en una configuración
paralela. Supongamos que la NASA quiere una probabilidad no menor que
0.99999 de que el transbordador espacial entre en órbita alrededor de la
Tierra con éxito. ¿Cúantos motores se deben configurar en paralelo para
que se consiga dicha fiabilidad si se sabe que la probabilidad de que un
motor funcione adecuadamente es 0.95? Supongamos que los motores
funcionan de manera independiente los unos con los otros.

\hypertarget{soluciuxf3n-12}{%
\subsubsection{Solución}\label{soluciuxf3n-12}}

Definimos el suceso: \(E_1\): éxito del motor\(i\)

\[
P(E_1) = 0.95
\]

Nos preguntan cuántos motores hay que poner para que la probabilidad de
que al menos 1 tenga éxito sea como mínimo 0.99999, que es igual que
decir que buscamos la configuración que con un mínimo de un 0.99999 de
probabilidad no va a tener ningún motor que falle:

\[
P(E_1 \cup E\_2 \cup \dots) = 0.99999 = 1 - P(E_1^c \cap E_2^c \cap \dots) = 1 - 0.05^n
\]

Resolvemos la inecuación:

\[
1 - 0.05^n \geq 0.99999 \\
0.05^n \geq 0.00001 \\
n \geq log_{0.05}(0.00001) \\
n \geq 3.842109 \\
n = 4
\]

\hypertarget{ejercicios-de-independencia-de-sucesos}{%
\section{Ejercicios de Independencia de
sucesos}\label{ejercicios-de-independencia-de-sucesos}}

\hypertarget{problema-1}{%
\subsection{Problema 1}\label{problema-1}}

Una moneda no trucada se lanza al aire 2 veces Consideremos los
siguientes sucesos:

\begin{verbatim}
* A: Sale una cara en la primera tirada.
* B: Sale una cara en la segunda tirada.

¿Son los sucesos A y B independientes?
\end{verbatim}

\hypertarget{soluciuxf3n-13}{%
\subsubsection{Solución}\label{soluciuxf3n-13}}

Sí

\hypertarget{problema-2}{%
\subsection{Problema 2}\label{problema-2}}

Una urna contiene 4 bolas numerades con los números 1, 2, 3 y 4,
respectivamente. Se extraen dos bolas sin reposición. Sea A el suceso
que la primera bola extraida tenga un 1 marcado y sea B el suceso que la
segunda bola extraida tenga un 1 marcado.

\begin{verbatim}
*  a) ¿Se puede decir que A y B son independientes?
*  b) ¿Y si el experimento fuera con reposición?
\end{verbatim}

\hypertarget{soluciuxf3n-14}{%
\subsubsection{Solución}\label{soluciuxf3n-14}}

En el primer caso no son independientes y en el segundo caso sí.

\hypertarget{problema-2-1}{%
\subsection{Problema 2}\label{problema-2-1}}

Sea \(\Omega\) un espacio muestral y \(A,B,C\) tres sucesos. Probad que

\begin{itemize}
\tightlist
\item
  \(A\) y \(B\) son independientes, también lo son \(A\) y \(B^c\)
\item
  Si \(A,B,C\) son independientes, también lo son \(A,B\) y \(C^c\)
\end{itemize}

\hypertarget{soluciuxf3n-15}{%
\subsection{Solución}\label{soluciuxf3n-15}}

Si los sucesos \(A\) y \(B\) son independientes tenemos que \$P(A
\cap B) = P(A)\cdot P(B).

Para que los sucesos \(A\) y \(B^c\) sean independientes se tiene que
cumplir que \(P(A \cap B^c) = P(A) \cdot P(B^c)\), operemos

\[
P(A \cap B^c) = P(A) - P(A \cap B) = P(A) \cdot (1 - P(B)) = P(A) \cdot P(B^c)
\]

por lo tanto \(A\) y \(B^c\) son sucesos independientes.

\begin{verbatim}
- Si $A,B,C$ son independientes, también lo son $A,B$ y $C^c$
\end{verbatim}

Si \(A,B,C\) son independientes significa que:\$P(A \cap B \cap C) =
P(A) \cdot P(B) \cdot P(C) \$ , \$P(A\cap B)=P(A) Y queremos comprobar
que: \[
P(A \cap B \cap C^c) = P(A) \cdot P(B) \cdot P(C^c)
\] Sabemos que: \[
P(A \cap B \cap C^c) = P(A \cap B) - P(A \cap B \cap C) = \\
P(A) \cdot P(B) - P(A)\cdot P(B) \cdot P(C) = \\
P(A) \cdot P(B) \cdot (1 - P(C)) = \\
P(A) \cdot P(B) \cdot P(C^c)
\] - ¿Es cierto que si \(A,B,C\) son independientes, también lo son
\(A,B^c\) y \(C^c\)? ¿Y \(A^c, B^c\) y \(C^c\)? En caso de que la
respuesta sea negativa, dad contraejemplos donde la propiedad falle.

Si: \[
P(A \cap B \cap C) = P(A) \cdot P(B) \cdot P(C)
\] Queremos comprobar si: \[
P(A \cap B^c \cap C^c) = P(A) - P(A \cap B) - P(A \cap C) + P(A \cap B \cap C) = \\
P(A) - P(A) \cdot P(B) - P(A) \cdot P(C) + P(A) \cdot P(B) \cdot P(C) = \\
P(A) \cdot (1 - P(B) - P(C) + P(B)\cdot P(C)) = \\
P(A) \cdot (1 - P(B) - P(C) \cdot (1 + P(B))) = \\
P(A) \cdot (1 - P(B) + P(C) \cdot (-1 - P(B))) = \\
P(A) \cdot (P(B^c) + P(C) \cdot -P(B^c)) = \\
P(A) \cdot (P(B^c) \cdot (1- P(C))) = \\
P(A) \cdot P(B^c) \cdot (P(C^c))
\]

\begin{enumerate}
\def\labelenumi{\arabic{enumi}.}
\setcounter{enumi}{3}
\item
  Dos empresas \(A\) y \(B\) fabrican el mismo producto. La empresa
  \(A\) tiene un \(2\%\) de productos defectuosos mientras que la
  empresa \(B\) tiene un \(1\%\). Un cliente recibe un pedido de una de
  las empresas (no sabe cuál) y comprueba que la primera pieza funciona.
  Si suponemos que el estado de las piezas de cada empresa es
  independiente, ¿cuál es la probabilidad de que la segunda pieza que
  pruebe sea buena? Comprobad que el estado de las dos piezas no es
  independiente, pero en cambio es condicionalmente independiente dada
  la empresa que las fabrica.
\item
  Encuentra un ejemplo de tres sucesos \(A,B,C\) tales que \(A\) y \(B\)
  sean independientes, pero en cambio no sean condicionalmente
  independientes dado \(C\).
\end{enumerate}

\end{document}
