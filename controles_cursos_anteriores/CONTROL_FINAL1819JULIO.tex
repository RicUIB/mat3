\documentclass[12pt]{article}\usepackage[]{graphicx}\usepackage[]{color}
% maxwidth is the original width if it is less than linewidth
% otherwise use linewidth (to make sure the graphics do not exceed the margin)
\makeatletter
\def\maxwidth{ %
  \ifdim\Gin@nat@width>\linewidth
    \linewidth
  \else
    \Gin@nat@width
  \fi
}
\makeatother

\definecolor{fgcolor}{rgb}{0.345, 0.345, 0.345}
\newcommand{\hlnum}[1]{\textcolor[rgb]{0.686,0.059,0.569}{#1}}%
\newcommand{\hlstr}[1]{\textcolor[rgb]{0.192,0.494,0.8}{#1}}%
\newcommand{\hlcom}[1]{\textcolor[rgb]{0.678,0.584,0.686}{\textit{#1}}}%
\newcommand{\hlopt}[1]{\textcolor[rgb]{0,0,0}{#1}}%
\newcommand{\hlstd}[1]{\textcolor[rgb]{0.345,0.345,0.345}{#1}}%
\newcommand{\hlkwa}[1]{\textcolor[rgb]{0.161,0.373,0.58}{\textbf{#1}}}%
\newcommand{\hlkwb}[1]{\textcolor[rgb]{0.69,0.353,0.396}{#1}}%
\newcommand{\hlkwc}[1]{\textcolor[rgb]{0.333,0.667,0.333}{#1}}%
\newcommand{\hlkwd}[1]{\textcolor[rgb]{0.737,0.353,0.396}{\textbf{#1}}}%
\let\hlipl\hlkwb

\usepackage{framed}
\makeatletter
\newenvironment{kframe}{%
 \def\at@end@of@kframe{}%
 \ifinner\ifhmode%
  \def\at@end@of@kframe{\end{minipage}}%
  \begin{minipage}{\columnwidth}%
 \fi\fi%
 \def\FrameCommand##1{\hskip\@totalleftmargin \hskip-\fboxsep
 \colorbox{shadecolor}{##1}\hskip-\fboxsep
     % There is no \\@totalrightmargin, so:
     \hskip-\linewidth \hskip-\@totalleftmargin \hskip\columnwidth}%
 \MakeFramed {\advance\hsize-\width
   \@totalleftmargin\z@ \linewidth\hsize
   \@setminipage}}%
 {\par\unskip\endMakeFramed%
 \at@end@of@kframe}
\makeatother

\definecolor{shadecolor}{rgb}{.97, .97, .97}
\definecolor{messagecolor}{rgb}{0, 0, 0}
\definecolor{warningcolor}{rgb}{1, 0, 1}
\definecolor{errorcolor}{rgb}{1, 0, 0}
\newenvironment{knitrout}{}{} % an empty environment to be redefined in TeX

\usepackage{alltt}
\usepackage{amsfonts,amssymb,amsmath,amsthm,graphicx,enumerate}
\usepackage[utf8]{inputenc}
\usepackage[T1]{fontenc}        
\usepackage[spanish]{babel}
\decimalpoint
\advance\hoffset by -0.9in
\advance\textwidth by 1.8in
\advance\voffset by -1in
\advance\textheight by 2in
\parskip= 1 ex
\parindent = 10pt
\baselineskip= 13pt
\newcommand{\red}[1]{\textcolor{red}{#1}}


\renewcommand{\leq}{\leqslant}
\renewcommand{\geq}{\geqslant}

\newcounter{problemes}
\newcounter{punts} \def\thepunts{\arabic{punts}}
\def\probl{\addtocounter{problemes}{1} \setcounter{punts}{0}
\medskip\noindent{\bf \theproblemes) }}
\def\punt{\addtocounter{punts}{1} \smallskip{\emph{\thepunts) }}}

\newcommand{\novapart}{\noindent\hrulefill}
\newcommand{\VV}{\textbf{\Large \checkmark}}
\newcommand{\coment}[1]{\noindent{\footnotesize\textbf{Comentario}: #1\par}}
\newcommand{\sol}[1]{{\footnotesize #1\par}}

\renewcommand{\VV}{}
\renewcommand{\sol}[1]{}
\renewcommand{\coment}[1]{}


\pagestyle{empty}
\IfFileExists{upquote.sty}{\usepackage{upquote}}{}
\begin{document}
%\SweaveOpts{concordance=TRUE}
%1
\noindent\emph{Nombre:}\hfill\hfill\hfill\hfill\hfill\hfill\hfill\ \emph{Grupo:}\hfill \vspace*{-2ex}

\begin{center}
\textsc{Matemáticas III. GINF. Control Final julio  2018-2019.}
\end{center}

\setcounter{problemes}{0}

\probl  Consideremos los siguientes sucesos $A$ y $B$ tales que 
$P(A\cup B)=0.8$, $P(A-B)=0.4$ y $P(B-A)=0.3$. Calcular $P(A\cap B)$, 
si es  posible.(\textbf{0.5 puntos.}).

\probl  Consideremos los siguientes sucesos $A$, $B$ y $C$ tales que
$P(A|B)=0.4$, $P(A|C)=0.7$, $P(B)=1-P(C)=0.2$. Calcular $P(A)$ y $P(C|A)$.
(\textbf{0.5 puntos.}).

\probl Consideremos la siguiente muestra $$4,-1,3,-2,2,1,-3,-4$$ estimar el error estándar de la media muestral (\textbf{0.5 puntos.}).

\probl  Sea $X$ una v.a. gaussiana con $\mu=7$ y $\sigma=3$. Se toma una muestra aleatoria simple de la variable $X$ de tamaño $n=25$. 
Calcular $P(6<\overline{X}<8)$  (\textbf{0.5 puntos.}).

\probl La probabilidad de que un cierto jugador de baloncesto enceste un tiro libre
es de $p=0.85$. Si durante un partido este jugador lanza 20 tiros libres,
contestar a las siguientes preguntas:
\begin{enumerate}[a)]
\item Sea $X$ la variable aleatoria que cuenta el número de tiros libres fallados.
Modelar $X$ con una v.a. de las estudiadas en clase, justificando la decisión.(\textbf{0.5 puntos.})
\item ¿Cúal es la probabilidad de que no falle ninguno?.(\textbf{0.5 puntos.})
\item ¿Cúal es la probabilidad de que falle más de 2 y menos de 5?(\textbf{0.5 puntos.})
\item ¿Cúal es el número esperado de tiros fallados?(\textbf{0.5 puntos.})
\end{enumerate}


\probl
Durante 120 días una persona observa el número de correos de SPAM recibidos
cada día. Los resultados se muestran en la siguiente tabla:


\begin{center}
\begin{tabular}{|c|cccccc|}
\hline
Número correos SPAM & 2 o menos & 3 & 4 & 5 & 6 & 7 o más\\ \hline
Número días & 10 & 20 & 21 & 23 & 20 & 26\\ \hline
\end{tabular}
\end{center}

\noindent
El número medio de correos diarios de SPAM durante este periodo
fue de 5. Deseamos contrastar la hipótesis nula de que la distribución del número
diario de mensajes de SPAM sigue una ley de Poisson. Para ello utilizaremos
un test de bondad de ajuste de $\chi^2$.

Para realizar el test necesitamos conocer las frecuencias observadas y las esperadas.
Los valores se muestran en la siguiente tabla:


\begin{center}
\begin{tabular}{|c|cccccc|}
\hline
Valores & 2 o menos & 3 & 4 & 5 & 6 & 7 o más\\ \hline
Frec. observada ($o_i$) & 10 & 20 & 21 & 23 & 20 & 26 \\ \hline
Frec. esperada ($e_i$)& 14.95824 & X & 21.05608 & X & 17.54674 & 28.53798 \\ \hline
\end{tabular}
\end{center}

Por otra parte, al realizar el test utilizando R obtenemos el siguiente 
resultado:

\begin{verbatim}
	Chi-squared test for given probabilities

data:  freq.obs
X-squared = 2.9828, df = X, p-value = XXXXX
\end{verbatim}

Se pide:
\begin{enumerate}[a)]
\item Completar los valores que faltan (marcados con X) en la tabla de valores esperados. (\textbf{1 punto.})
\item Calcular el p-valor del contraste y decidir si se puede aceptar
o no que los datos observados siguen una distribución de Poisson. (\textbf{1 punto.})
\end{enumerate}

% > frec.obs=c(10, 20, 21, 23, 20, 26)
% > probs=c(ppois(2, 5), dpois(3:6, 5), 1-ppois(6, 5))
% > n=120
% > frec.esp=n*probs
% > frec.esp
% [1] 14.95824 16.84487 21.05608 21.05608 17.54674 28.53798
% > chi2=sum((frec.obs-frec.esp)^2 / frec.esp)
% > chi2
% [1] 2.982816
% > pvalor=1-pchisq(chi2, 4)
% > pvalor
% [1] 0.5607052
% 
% > chisq.test(frec.obs, p=probs)
% 
% 	Chi-squared test for given probabilities
% 
% data:  frec.obs
% X-squared = 2.9828, df = 5, p-value = 0.7026



\probl Siguiendo con el ejemplo anterior, supongamos que 
la persona escoge aleatoriamente 200 de los mensajes recibidos 
durante un mes y que un $10\%$ de los cuales son SPAM.
Al cabo de 6 meses vuelve a realizar el estudio, para lo que elige
aleatoriamente 250 de los mensajes recibidos durante el último
mes, de los cuales un $15\%$ son SPAM. Consideramos que las dos
muestras observadas son independientes.

Queremos contrastar la hipótesis de que el porcentaje de SPAM es el mismo en los
dos periodos de tiempo observados, respecto a de que es diferente:

\begin{enumerate}[a)]
\item Escribir explícitamente las hipótesis nula y alternativa y calcular el estadístico de contraste adecuado  (\textbf{0.5 puntos.})
\item Resolver el contraste calculando el $p$-valor.  (\textbf{1 punto.})
\item Calcular el intervalo de confianza para la diferencia de porcentajes del 95\% asociado al contraste de hipótesis. (\textbf{1 punto.})
\end{enumerate}


\probl Durante 6 días consecutivos se mide el número medio de clicks por hora en los anuncios de una página web.
Los valores obtenidos (\texttt{data}) son:

\begin{verbatim}
> data=c(17.62, 18.56, 21.52, 17.69, 20.73, 17.50)
> mean(data)
[1] 18.93667
> sd(data)
[1] 1.754043
\end{verbatim}


El diseñador de la página web afirma que con su diseño el número medio de clicks es de 20 por hora. ¿Podemos
aceptar esta afirmación con un nivel de significación del 5\%,
suponiendo que la distribución de los valores sigue una ley gaussiana?
(\textbf{1.5 puntos.})



\end{document}

