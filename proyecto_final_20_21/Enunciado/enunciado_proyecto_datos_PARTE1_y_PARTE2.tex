% Options for packages loaded elsewhere
\PassOptionsToPackage{unicode}{hyperref}
\PassOptionsToPackage{hyphens}{url}
\PassOptionsToPackage{dvipsnames,svgnames*,x11names*}{xcolor}
%
\documentclass[
]{article}
\usepackage{lmodern}
\usepackage{amssymb,amsmath}
\usepackage{ifxetex,ifluatex}
\ifnum 0\ifxetex 1\fi\ifluatex 1\fi=0 % if pdftex
  \usepackage[T1]{fontenc}
  \usepackage[utf8]{inputenc}
  \usepackage{textcomp} % provide euro and other symbols
\else % if luatex or xetex
  \usepackage{unicode-math}
  \defaultfontfeatures{Scale=MatchLowercase}
  \defaultfontfeatures[\rmfamily]{Ligatures=TeX,Scale=1}
\fi
% Use upquote if available, for straight quotes in verbatim environments
\IfFileExists{upquote.sty}{\usepackage{upquote}}{}
\IfFileExists{microtype.sty}{% use microtype if available
  \usepackage[]{microtype}
  \UseMicrotypeSet[protrusion]{basicmath} % disable protrusion for tt fonts
}{}
\makeatletter
\@ifundefined{KOMAClassName}{% if non-KOMA class
  \IfFileExists{parskip.sty}{%
    \usepackage{parskip}
  }{% else
    \setlength{\parindent}{0pt}
    \setlength{\parskip}{6pt plus 2pt minus 1pt}}
}{% if KOMA class
  \KOMAoptions{parskip=half}}
\makeatother
\usepackage{xcolor}
\IfFileExists{xurl.sty}{\usepackage{xurl}}{} % add URL line breaks if available
\IfFileExists{bookmark.sty}{\usepackage{bookmark}}{\usepackage{hyperref}}
\hypersetup{
  pdftitle={Proyecto asignatura Estadística (MAT3) GIN2 Grupos 1 y 3: PARTE 1 y PARTE 2. CUrso 2020-2021},
  pdfauthor={Poned los nombres de los autores},
  colorlinks=true,
  linkcolor=Maroon,
  filecolor=Maroon,
  citecolor=Blue,
  urlcolor=blue,
  pdfcreator={LaTeX via pandoc}}
\urlstyle{same} % disable monospaced font for URLs
\usepackage[margin=1in]{geometry}
\usepackage{graphicx}
\makeatletter
\def\maxwidth{\ifdim\Gin@nat@width>\linewidth\linewidth\else\Gin@nat@width\fi}
\def\maxheight{\ifdim\Gin@nat@height>\textheight\textheight\else\Gin@nat@height\fi}
\makeatother
% Scale images if necessary, so that they will not overflow the page
% margins by default, and it is still possible to overwrite the defaults
% using explicit options in \includegraphics[width, height, ...]{}
\setkeys{Gin}{width=\maxwidth,height=\maxheight,keepaspectratio}
% Set default figure placement to htbp
\makeatletter
\def\fps@figure{htbp}
\makeatother
\setlength{\emergencystretch}{3em} % prevent overfull lines
\providecommand{\tightlist}{%
  \setlength{\itemsep}{0pt}\setlength{\parskip}{0pt}}
\setcounter{secnumdepth}{5}
\renewcommand{\contentsname}{Contenidos}

\title{Proyecto asignatura Estadística (MAT3) GIN2 Grupos 1 y 3: PARTE 1
y PARTE 2. CUrso 2020-2021}
\author{Poned los nombres de los autores}
\date{}

\begin{document}
\maketitle

{
\hypersetup{linkcolor=blue}
\setcounter{tocdepth}{2}
\tableofcontents
}
\hypertarget{parte-1-estaduxedstica-descriptiva-1-punto-sobre-10-de-la-nota-final.-entregad-por-aula-digital-antes-del-21-de-abril}{%
\section{Parte 1: Estadística Descriptiva (1 punto sobre 10 de la nota
final. Entregad por Aula Digital antes del 21 de
abril)}\label{parte-1-estaduxedstica-descriptiva-1-punto-sobre-10-de-la-nota-final.-entregad-por-aula-digital-antes-del-21-de-abril}}

La proyecto se entrega en \textbf{grupos de 4 persones}. Cada grupo
tendrá asignado un \textbf{nombre de ciudad} que podréis consultar en la
página de la asignatura en Aula Digital. La práctica consiste en obtener
los datos de la ciudad asignada, correspondientes a \textbf{febrero de
2020}, de la \href{http://insideairbnb.com/get-the-data.html}{web de
Airbnb} y redactar un informe utilizando Rmarkdown respondiendo a las
siguientes cuestiones

\begin{enumerate}
\def\labelenumi{\arabic{enumi}.}
\item
  Cargad en un \texttt{dataframe} los datos del fichero
  \texttt{listings.csv} (descomprimido a partir de
  \texttt{listings.csv.gz}). Esta tabla de datos contiene datos sobre
  muchas variables:

  \begin{itemize}
  \tightlist
  \item
    \texttt{neighbourhood\ property\_type}
  \item
    \texttt{accommodates}
  \item
    \texttt{beds}
  \item
    \texttt{bedrooms}
  \item
    \texttt{bathrooms}
  \item
    \texttt{price}
  \item
    \texttt{security\_deposit}
  \item
    \texttt{minimum\_nights}
  \item
    \texttt{review\_scores\_rating}
  \item
    \texttt{number\_of\_reviews}
  \item
    \texttt{host\_response\_time}
  \item
    \texttt{requires\_license}
  \item
    \texttt{review\_scores\_cleanliness}
  \item
    \texttt{cleaning\_fee}
  \item
    \texttt{etc.}~\\
    Construid un nuevo \texttt{data\ frame} que contenga solo la
    información relativa a como mínimo 3 variables no numéricas y 7
    variables numéricas. Hay que \textbf{escoger forma obligatoria las
    variables} \texttt{neighbourhood} y \texttt{price}. Las variables
    relativas a precioso (por ejemplo, \texttt{price} y
    \texttt{security\_deposit}) se consideran numéricas, pero primero
    deben convertirse a valores numéricos eliminando los símbolos
    \texttt{\$} o \texttt{€}; por ejemplo utilizando la función
    \texttt{gsub("pattern={[}\textbackslash{}\textbackslash{}\${]}",replacement="",x="\$10,000")};
    haced (\texttt{help(gsub)})
  \end{itemize}
\item
  Renombrar las columnas del \texttt{dataframe} con nombres en
  castellano o catalán.
\item
  Para las variables numéricas, calcular los siguientes estadísticos y
  mostrar en una tabla los siguiente valores: cantidad de valores no
  válidos, mínimo, máximo, media, varianza, cuartiles y mediana.
\item
  Para las variables no numéricas, generad las tablas de frecuencias
  absolutas de cada uno de sus valores.
\item
  Dibujar diagramas de cajas(\textbf{boxplots }) e histogramas de todas
  las variables numéricas, mostrando un mínimo de 2 \textbf{boxplot} por
  fila. Ajustad la altura de los gráficos para que no queden demasiado
  pequeños.
\item
  Dibuja un diagrama de tarta para una de las variables no numéricas
  agrupando como ``Otros'' los valores que representen un porcentaje
  inferior al 1\% del total.
\item
  Calcular el valor medio de alguna de las variables numéricas según el
  barrio, de menor a mayor y sin tener en cuenta nombres de Barrio
  incorrectas ( \texttt{""} o \texttt{"N/A"}).
\item
  Dibuja un \textbf{boxplot} de la variable precio, para los 5 barrios
  más caros (precio medio más alto) y los 5 barrios más baratos (precio
  medio más bajo), en un mismo diagrama. Los \textbf{boxplots} deben
  indicar también el valor medio de los datos.
\item
  Para 4 de las variables numéricas dibujar los diagramas de dispersión
  dos a dos, con colores diferentes para cada barrio (
  ``neighbourhood'').
\item
  Por las mismas variables elegidas en el apartado anterior calcular los
  coeficientes de correlación dos a dos de las variables, sin tener en
  cuenta los valores \texttt{NA}.
\item
  Selecciona dos variables numéricas, y para cada una de ellas organiza
  sus valores en un máximo de 5 intervalos con ala función \texttt{cut}.
\item
  A partir de los datos en intervalos obtenidos en el apartado anterior
  construid una tabla de contingencia de las dos variables y dibuja el
  diagrama de mosaico asociado a la mesa.
\end{enumerate}

Se valorará la claridad y los comentarios de los resultados obtenidos.
Si se detectan ** trabajos copiados \textbf{ quedarán } suspendidos
todos los alumnos implicados **. Todas las preguntas se pueden contestar
a partir de la información de los manuales de
\href{https://aprender-uib.github.io/AprendeR1/}{AprendeR1},
\href{https://aprender-uib.github.io/AprendeR2/}{AprendeR2} y
concretamente
\href{https://aprender-uib.github.io/AprendeR2/chap-estmult.html}{AprendeR2
tema 8}, a parte de los enlaces proporcionados en este documento, y, en
general, haciendo búsquedas en internet. También puede basar su informe
en los ejemplos publicados en la web de la asignatura, Además, puede
consultar dudas con el resto de los compañeros de curso a través del
Foro de la asignatura a Aula Digital o en discord.

El documento en formato .Rmd y el informe en .html o .pdf se debe en la
actividad correspondiente del \textbf{Aula Digital antes del 21 de
abril}.

\hypertarget{parte-2-estaduxedstica-inferencial-1-punto-sobre-10-de-la-nota-final.-entregad-por-aula-digital-antes-del-15-de-junio}{%
\section{Parte 2: Estadística Inferencial (1 punto sobre 10 de la nota
final. Entregad por Aula Digital antes del 15 de junio
)}\label{parte-2-estaduxedstica-inferencial-1-punto-sobre-10-de-la-nota-final.-entregad-por-aula-digital-antes-del-15-de-junio}}

Supongamos que los datos de la ciudad que ha sido asignada a cada grupo
corresponden a una muestra aleatoria simple de todas las viviendas que
se podrían alquilar en la ciudad. Utilizando esta muestra se pide:

\begin{enumerate}
\def\labelenumi{\arabic{enumi}.}
\item
  Calcular una estimación puntual de la media para la variable
  \texttt{price} y el error estándar del estimador.
\item
  Calcular un intervalo de confianza, al nivel de confianza del 95\%,
  para la variable \texttt{price}.
\item
  Calcular un intervalo de confianza, al nivel de confianza del 99\%,
  para la proporción de viviendas que tienen un
  \texttt{review\_scores\_rating} inferior a 95\%.
\item
  Supongamos que un responsable de Airbnb asegura que la media de los
  valores de \texttt{review\_scores\_rating} de las viviendas de su
  portal es superior a 95. Contrastad esta hipótesis.
\item
  Calcular el intervalo de confianza, con un nivel de confianza del
  95\%, asociado al contraste del ejercicio anterior.
\item
  Considera ahora los datos de \texttt{price} para la ciudad de New York
  del mes de febrero de 2020 ( están en
  \url{http://insideairbnb.com/get-the-data.html}, y debe pulsar en
  `show archived fecha ). Compararemos los valores de esta variable con
  los correspondientes a la ciudad que tiene asignada. Haga un contraste
  de hipótesis para decidir si las desviaciones típicas de los precios
  de las dos ciudades son iguales o diferentes. Considera que las
  distribuciones de los valores de precio en las poblaciones son
  normales.
\item
  A partir de los resultados del apartado anterior contratad la
  hipótesis de que los precios medios en las dos ciudades son iguales.
\item
  Utilice el test de Kolmogorov-Smirnov-Lilliefors para confirmar o
  rechazar la hipótesis de que la distribución de los valores de la
  variable \texttt{price} es normal, decidid el resultado del contraste
  con el \(p\)-valor.
\item
  Analizad la dependencia entre las variables \texttt{Price} y
  \texttt{review\_scores\_rating} de la ciudad que tiene asignada.
  Seguid los siguientes pasos:

  \begin{itemize}
  \item
    \begin{enumerate}
    \def\labelenumii{\alph{enumii})}
    \tightlist
    \item
      Seleccione del data frame las muestras que tienen valores
      diferentes de NA por las dos variables.
    \end{enumerate}
  \item
    \begin{enumerate}
    \def\labelenumii{\alph{enumii})}
    \setcounter{enumii}{1}
    \tightlist
    \item
      A continuación agrupau los valores de cada variable utilizando los
      intervalos siguientes: \([ \min, Q_1), [Q_1, Q_2), [Q_2, Q_3)\) y
      \([Q_3, \max]\). Los valores \(\min\) y \(\max\) son el mínimo y
      el máximo de la variable, respectivamente. Mientras que \(Q_1\),
      \(Q_2\) y \(Q_3\)· representan los cuartiles primero, segundo
      (mediana) y tercero. Si los valor mínimo y máximo de algún
      intervalo son iguales elimine este intervalo.
    \end{enumerate}
  \item
    \begin{enumerate}
    \def\labelenumii{\alph{enumii})}
    \setcounter{enumii}{2}
    \tightlist
    \item
      Organizad los datos agrupados en intervalos en una tabla de
      contingencia \texttt{Price} versus \texttt{review\_scores\_rating}
      .
    \end{enumerate}
  \item
    \begin{enumerate}
    \def\labelenumii{\alph{enumii})}
    \setcounter{enumii}{3}
    \tightlist
    \item
      A partir de esta tabla haced un test \(\chi^2\) de independencia
      para determinar si las dos variables son independientes, con un
      nivel de significación del 0.05.
    \end{enumerate}
  \end{itemize}
\end{enumerate}

Comentarios:

\begin{itemize}
\item
  Para hacer los cálculos solicitados en los apartados anterior se deben
  eliminar los valores no disponibles (\texttt{NA}) de las variables.
\item
  Siempre que sea posible se deben utilizar las funciones de R
  explicadas en clase para resolver los ejercicios.
\item
  Debe redactar un documento utilizando Rmarkdown con las respuestas a
  estas preguntas y que incluya el código R utilizado. También debe
  generar (Knit) una versión HTML del documento.
\end{itemize}

El documento, en formato .Rmd y .html o .pdf , se debe \textbf{entregar
a Aula Digital antes del 15 de junio}.

\end{document}
