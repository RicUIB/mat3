% Options for packages loaded elsewhere
\PassOptionsToPackage{unicode}{hyperref}
\PassOptionsToPackage{hyphens}{url}
%
\documentclass[
]{article}
\usepackage{lmodern}
\usepackage{amssymb,amsmath}
\usepackage{ifxetex,ifluatex}
\ifnum 0\ifxetex 1\fi\ifluatex 1\fi=0 % if pdftex
  \usepackage[T1]{fontenc}
  \usepackage[utf8]{inputenc}
  \usepackage{textcomp} % provide euro and other symbols
\else % if luatex or xetex
  \usepackage{unicode-math}
  \defaultfontfeatures{Scale=MatchLowercase}
  \defaultfontfeatures[\rmfamily]{Ligatures=TeX,Scale=1}
\fi
% Use upquote if available, for straight quotes in verbatim environments
\IfFileExists{upquote.sty}{\usepackage{upquote}}{}
\IfFileExists{microtype.sty}{% use microtype if available
  \usepackage[]{microtype}
  \UseMicrotypeSet[protrusion]{basicmath} % disable protrusion for tt fonts
}{}
\makeatletter
\@ifundefined{KOMAClassName}{% if non-KOMA class
  \IfFileExists{parskip.sty}{%
    \usepackage{parskip}
  }{% else
    \setlength{\parindent}{0pt}
    \setlength{\parskip}{6pt plus 2pt minus 1pt}}
}{% if KOMA class
  \KOMAoptions{parskip=half}}
\makeatother
\usepackage{xcolor}
\IfFileExists{xurl.sty}{\usepackage{xurl}}{} % add URL line breaks if available
\IfFileExists{bookmark.sty}{\usepackage{bookmark}}{\usepackage{hyperref}}
\hypersetup{
  pdftitle={Taller 1 problemas. MAT3 (estadística) GIN2 2020-2021 - Probabilidad, Variables Aleatorias, Distribuciones Notables.},
  pdfauthor={PONED LOS NOMBRES de grupo de autores},
  hidelinks,
  pdfcreator={LaTeX via pandoc}}
\urlstyle{same} % disable monospaced font for URLs
\usepackage[margin=1in]{geometry}
\usepackage{graphicx}
\makeatletter
\def\maxwidth{\ifdim\Gin@nat@width>\linewidth\linewidth\else\Gin@nat@width\fi}
\def\maxheight{\ifdim\Gin@nat@height>\textheight\textheight\else\Gin@nat@height\fi}
\makeatother
% Scale images if necessary, so that they will not overflow the page
% margins by default, and it is still possible to overwrite the defaults
% using explicit options in \includegraphics[width, height, ...]{}
\setkeys{Gin}{width=\maxwidth,height=\maxheight,keepaspectratio}
% Set default figure placement to htbp
\makeatletter
\def\fps@figure{htbp}
\makeatother
\setlength{\emergencystretch}{3em} % prevent overfull lines
\providecommand{\tightlist}{%
  \setlength{\itemsep}{0pt}\setlength{\parskip}{0pt}}
\setcounter{secnumdepth}{-\maxdimen} % remove section numbering

\title{Taller 1 problemas. MAT3 (estadística) GIN2 2020-2021 -
Probabilidad, Variables Aleatorias, Distribuciones Notables.}
\author{PONED LOS NOMBRES de grupo de autores}
\date{}

\begin{document}
\maketitle

\hypertarget{taller1-evaluable.-entrega-de-problemas}{%
\section{Taller1 evaluable. Entrega de
problemas}\label{taller1-evaluable.-entrega-de-problemas}}

Taller en grupo entregad las soluciones en ,Rmd y .html o .pdf. o si lo
hacedlas de forma manual y escanear el resultado, en un solo fichero.

\hypertarget{problema-1}{%
\subsection{Problema 1}\label{problema-1}}

Encuentra un ejemplo de tres sucesos \(A,B,C\) tales que \(A\) y \(B\)
sean independientes, pero en cambio no sean condicionalmente
independientes dado \(C\).

\hypertarget{soluciuxf3n}{%
\subsubsection{Solución}\label{soluciuxf3n}}

\hypertarget{problema-2}{%
\subsection{Problema 2}\label{problema-2}}

Consideremos la v.a. continua \(X\) que tiene por función de densidad
para a alguna constante \(\alpha\in \mathbb{R}\):

\[
f_X (t)=
\left\{\begin{array}{ll}
\alpha \cdot t^4, & \mbox{si } -1 < t <1,
 \\
0 & \mbox{ en otro caso}.
\end{array}\right.
\]

\begin{enumerate}
\def\labelenumi{\arabic{enumi}.}
\tightlist
\item
  Calculad \(\alpha\) para que \(f_X\) sea densidad y especificad su
  dominio \(D_X\).
\item
  Calculad la función de distribución de la v.a. \(X\);
  \(F_X(x)=P(X\leq x)\).
\item
  Calculad \(E(X)\) y \(Var(X)\).
\item
  Calcula en cuantil \(0.9\) de \(X\).
\end{enumerate}

\hypertarget{soluciuxf3n-1}{%
\subsubsection{Solución}\label{soluciuxf3n-1}}

\hypertarget{problema-3}{%
\subsection{Problema 3}\label{problema-3}}

Sea \(Y\) una variable discreta con función de probabilidad :

\[
P_Y(y)=
\left\{\begin{array}{ll}
\alpha\cdot\frac{1}{x^2} & \mbox{,si } x=2,-1,0,1,2,
 \\
0 & \mbox{ en otro caso}.
\end{array}\right. 
\]

\begin{enumerate}
\def\labelenumi{\arabic{enumi}.}
\tightlist
\item
  Hallad la función de distribución \(F_Y(y)=P(Y\leq Y)\).
\item
  Calculad \(E(Y)\) y \(Var(Y)\)
\item
  Calculad el cuantil 0.5 de de \(Y\)
\end{enumerate}

\hypertarget{soluciuxf3n-2}{%
\subsubsection{Solución}\label{soluciuxf3n-2}}

\hypertarget{problema-4}{%
\subsection{Problema 4}\label{problema-4}}

Tenemos un dado, bien equilibrado, de doce caras numeradas del 1 al 12
(\href{https://es.wikipedia.org/wiki/Dados_de_rol}{dodecaedro dados de
rol}).

\begin{itemize}
\item
  \begin{enumerate}
  \def\labelenumi{\alph{enumi})}
  \tightlist
  \item
    Calcular la función de probabilidad de la variables \(X=\) número de
    la cara superior del dado en un lanzamiento, calcular E(X) y
    \(Var(X)\).
  \end{enumerate}
\item
  \begin{enumerate}
  \def\labelenumi{\alph{enumi})}
  \setcounter{enumi}{1}
  \tightlist
  \item
    Calcular la función de distribución de \(X\) y el cuantil \(0.4\).
  \end{enumerate}
\item
  \begin{enumerate}
  \def\labelenumi{\alph{enumi})}
  \setcounter{enumi}{2}
  \tightlist
  \item
    Si \(Y\) es al v.a.que cuenta el número de veces que tiramos el dado
    hasta obtener el primer \(5\) calcular la función de distribución de
    \(Y\)
  \end{enumerate}
\item
  ¿Qué valor tienen \$E(X) \$ y \(Var(X)\).
\end{itemize}

\hypertarget{soluciuxf3n-3}{%
\subsubsection{Solución}\label{soluciuxf3n-3}}

\hypertarget{problema-5}{%
\subsection{Problema 5}\label{problema-5}}

La proporción de niños pelirrojos es 1 cada 100. En una ciudad se
produjeron 500 nacimientos (independientes) nacimientos en 2020, modelad
mediante una distribución binomial la variable \(X\) = número de niños
pelirrojos nacidos entre los 500 niños. Utilizad R para calcular de
forma exacta

\begin{itemize}
\item
  \begin{enumerate}
  \def\labelenumi{\alph{enumi})}
  \tightlist
  \item
    La probabilidad de que ninguno de los nacidos ese año sea pelirrojo.
  \end{enumerate}
\item
  \begin{enumerate}
  \def\labelenumi{\alph{enumi})}
  \setcounter{enumi}{1}
  \tightlist
  \item
    La probabilidad de que nazcan más de 2 niños pelirrojos
  \end{enumerate}
\item
  \begin{enumerate}
  \def\labelenumi{\alph{enumi})}
  \setcounter{enumi}{2}
  \tightlist
  \item
    Repetir los´cálculos con R utilizando una aproximación Poisson
  \end{enumerate}
\end{itemize}

\hypertarget{soluciuxf3n-4}{%
\subsubsection{Solución}\label{soluciuxf3n-4}}

\hypertarget{problema-6}{%
\subsection{Problema 6}\label{problema-6}}

Las consultas a una base dato llegan a un ritmo de medio \(\lambda=5\)
peticiones por segundo. Sabemos que el nombre de peticiones que llegan
en un segundo es una variable aleatoria que aproximadamente tienen una
distribución de Poisson.

\begin{itemize}
\item
  \begin{enumerate}
  \def\labelenumi{\alph{enumi})}
  \tightlist
  \item
    Calcular la probabilidad que lleguen más de 10 peticiones en un 3
    segundos.
  \end{enumerate}
\item
  \begin{enumerate}
  \def\labelenumi{\alph{enumi})}
  \setcounter{enumi}{1}
  \tightlist
  \item
    Calcular que entre una consulta y la siguiente pasen \(0.5\)
    segundos.
  \end{enumerate}
\item
  \begin{enumerate}
  \def\labelenumi{\alph{enumi})}
  \setcounter{enumi}{2}
  \tightlist
  \item
    Calcular el cuantil 0.5 de \(X_{t=10}\) numero de peticiones en 10
    segundos utilizad R
  \end{enumerate}
\end{itemize}

\hypertarget{soluciuxf3n-5}{%
\subsubsection{Solución}\label{soluciuxf3n-5}}

\hypertarget{problema-7}{%
\subsection{Problema 7}\label{problema-7}}

Tenemos que elegir entre dos programas (Prog1 y Prog2), el objetivo es
elegir el programa más rápido en tiempo de respuesta en nuestro cluster
de ordenadores. El tiempo de ejecución del Prog1 se ha modelado según
una \(N(\mu_1=100, \sigma_1=300)\) (la probabilidad de un tiempo de
ejecución negativo es despreciable) y en Prog2 según una
\(N(\mu_2=90, \sigma_2=300)\). Utilizad R para el cálculo final de las
probabilidades de la normal.

\begin{itemize}
\item
  \begin{enumerate}
  \def\labelenumi{\alph{enumi})}
  \tightlist
  \item
    ¿Qué Programa elegimos si queremos que el el 90\% de los casos el
    tiempo de respuesta sea menor ?
  \end{enumerate}
\item
  \begin{enumerate}
  \def\labelenumi{\alph{enumi})}
  \setcounter{enumi}{1}
  \tightlist
  \item
    Calcular la probabilidad de que el tiempo de ejecución sea mayor que
    130 para cada algoritmos.
  \end{enumerate}
\end{itemize}

\hypertarget{problema-8}{%
\subsection{Problema 8}\label{problema-8}}

En la NBA el
\href{José\%20Calderón}{https://es.wikipedia.org/wiki/Jos\%C3\%A9\_Manuel\_Calder\%C3\%B3n}
fue en la temporada
\href{https://es.wikipedia.org/wiki/L\%C3\%ADderes_en_porcentaje_de_tiros_libres_de_la_NBA}{2008-09
el jugador de baloncesto} {[}com mejor porcentaje tiros libres anotados
un 98.05\%.

Justificar los cálculos con notación matemática y haced el cálculo final
con R

\begin{itemize}
\item
  \begin{enumerate}
  \def\labelenumi{\alph{enumi})}
  \tightlist
  \item
    ¿Cual es el valor esperando y la varianza del número tiros hasta
    aceptar los 10 tiros libres?
  \end{enumerate}
\item
  \begin{enumerate}
  \def\labelenumi{\alph{enumi})}
  \setcounter{enumi}{1}
  \tightlist
  \item
    ¿Cuál es la probabilidad de que acierte al menos 40 tiros libres de
    forma consecutiva.
  \end{enumerate}
\item
  \begin{enumerate}
  \def\labelenumi{\alph{enumi})}
  \setcounter{enumi}{2}
  \tightlist
  \item
    ¿Cuál es la probabilidad de que haga una serie de 100 tiros hasta
    obtener el tercer fallo?
  \end{enumerate}
\end{itemize}

\end{document}
