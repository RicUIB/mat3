\documentclass[12pt]{article}\usepackage[]{graphicx}\usepackage[]{color}
% maxwidth is the original width if it is less than linewidth
% otherwise use linewidth (to make sure the graphics do not exceed the margin)
\makeatletter
\def\maxwidth{ %
  \ifdim\Gin@nat@width>\linewidth
    \linewidth
  \else
    \Gin@nat@width
  \fi
}
\makeatother

\definecolor{fgcolor}{rgb}{0.345, 0.345, 0.345}
\newcommand{\hlnum}[1]{\textcolor[rgb]{0.686,0.059,0.569}{#1}}%
\newcommand{\hlstr}[1]{\textcolor[rgb]{0.192,0.494,0.8}{#1}}%
\newcommand{\hlcom}[1]{\textcolor[rgb]{0.678,0.584,0.686}{\textit{#1}}}%
\newcommand{\hlopt}[1]{\textcolor[rgb]{0,0,0}{#1}}%
\newcommand{\hlstd}[1]{\textcolor[rgb]{0.345,0.345,0.345}{#1}}%
\newcommand{\hlkwa}[1]{\textcolor[rgb]{0.161,0.373,0.58}{\textbf{#1}}}%
\newcommand{\hlkwb}[1]{\textcolor[rgb]{0.69,0.353,0.396}{#1}}%
\newcommand{\hlkwc}[1]{\textcolor[rgb]{0.333,0.667,0.333}{#1}}%
\newcommand{\hlkwd}[1]{\textcolor[rgb]{0.737,0.353,0.396}{\textbf{#1}}}%
\let\hlipl\hlkwb

\usepackage{framed}
\makeatletter
\newenvironment{kframe}{%
 \def\at@end@of@kframe{}%
 \ifinner\ifhmode%
  \def\at@end@of@kframe{\end{minipage}}%
  \begin{minipage}{\columnwidth}%
 \fi\fi%
 \def\FrameCommand##1{\hskip\@totalleftmargin \hskip-\fboxsep
 \colorbox{shadecolor}{##1}\hskip-\fboxsep
     % There is no \\@totalrightmargin, so:
     \hskip-\linewidth \hskip-\@totalleftmargin \hskip\columnwidth}%
 \MakeFramed {\advance\hsize-\width
   \@totalleftmargin\z@ \linewidth\hsize
   \@setminipage}}%
 {\par\unskip\endMakeFramed%
 \at@end@of@kframe}
\makeatother

\definecolor{shadecolor}{rgb}{.97, .97, .97}
\definecolor{messagecolor}{rgb}{0, 0, 0}
\definecolor{warningcolor}{rgb}{1, 0, 1}
\definecolor{errorcolor}{rgb}{1, 0, 0}
\newenvironment{knitrout}{}{} % an empty environment to be redefined in TeX

\usepackage{alltt}
\usepackage{amsfonts,amssymb,amsmath,amsthm,graphicx,enumerate}
\usepackage[utf8]{inputenc}
\usepackage[T1]{fontenc}        
\usepackage[spanish]{babel}
\decimalpoint
\advance\hoffset by -0.9in
\advance\textwidth by 1.8in
\advance\voffset by -1in
\advance\textheight by 2in
\parskip= 1 ex
\parindent = 10pt
\baselineskip= 13pt
\newcommand{\red}[1]{\textcolor{red}{#1}}


\renewcommand{\leq}{\leqslant}
\renewcommand{\geq}{\geqslant}

\newcounter{problemes}
\newcounter{punts} \def\thepunts{\arabic{punts}}
\def\probl{\addtocounter{problemes}{1} \setcounter{punts}{0}
\medskip\noindent{\bf \theproblemes) }}
\def\punt{\addtocounter{punts}{1} \smallskip{\emph{\thepunts) }}}

\newcommand{\novapart}{\noindent\hrulefill}
\newcommand{\VV}{\textbf{\Large \checkmark}}
\newcommand{\coment}[1]{\noindent{\footnotesize\textbf{Comentario}: #1\par}}
\newcommand{\sol}[1]{{\footnotesize #1\par}}

\renewcommand{\VV}{}
\renewcommand{\sol}[1]{}
\renewcommand{\coment}[1]{}


\pagestyle{plain}
\IfFileExists{upquote.sty}{\usepackage{upquote}}{}
\begin{document}
%\SweaveOpts{concordance=TRUE}
%1
\noindent\emph{Nombre:}\hfill\hfill\hfill\hfill\hfill\hfill\hfill\ \emph{Grupo:}\hfill \vspace*{-2ex}

\begin{center}
\textsc{Matemáticas III. GIN2. Control Parcial abril  2020-2021.}
\end{center}

\setcounter{problemes}{0}

\probl  Consideremos los siguientes sucesos $A$, $B$ tales que
$P(A|B)=0.4$, $P(A|B^c)=0.7$, $P(B^c)=0.2$. Calcular $P(A)$ y $P(B|A)$.
(\textbf{1 punto}).

\textbf{Solución:}




$$P(A)=P(A|B)\cdot P(B)+P(A|B^c)\cdot P(B^c)=0.4\cdot (1-0.2)+ 0.5\cdot 0.2=0.46.$$

$$P(B|A)=\frac{P(A|B)\cdot P(B)}{P(A)}=\frac{0.4\cdot (1-0.2)}{0.46}=0.173913.$$


\probl \textbf{\textsc{PUNTO EXTRA EN ESTE EXAMEN}}. Tiramos 10 dados de parchís hasta obtener exactamente 5 cincos incluido ese último lanzamiento.
Sea $X$  el número de tiradas necesarias ¿Cuál es la distribución de $X$ su valor esperado y su varianza? (\textbf{1 punto}).

\textbf{Solución:}

Sea $Y$ el número de cincos obtenidos al tirar $n=10$ dados de 6 caras cop probabilidad de 5 $p_5=\frac{1}{6}$, samemos que $Y$ sigue una ley $B(n=10,p=p_5=\frac{1}{6})$.  Con R la probabilidad de obtener exactamente 5 cincos, es decir  $P(Y=5)$, es:

\begin{knitrout}
\definecolor{shadecolor}{rgb}{0.969, 0.969, 0.969}\color{fgcolor}\begin{kframe}
\begin{alltt}
\hlkwd{dbinom}\hlstd{(}\hlnum{5}\hlstd{,}\hlkwc{size}\hlstd{=}\hlnum{10}\hlstd{,}\hlkwc{prob}\hlstd{=}\hlnum{1}\hlopt{/}\hlnum{6}\hlstd{)}
\end{alltt}
\begin{verbatim}
## [1] 0.01302381
\end{verbatim}
\end{kframe}
\end{knitrout}

O tambien haciendo 

$$P(Y=5)=\frac{\mbox{Casos Favorables}}{\mbox{Casos Posibles}}
= \frac{{10 \choose 5}\cdot 5^5}{6^{10}},$$

con R
\begin{knitrout}
\definecolor{shadecolor}{rgb}{0.969, 0.969, 0.969}\color{fgcolor}\begin{kframe}
\begin{alltt}
\hlkwd{choose}\hlstd{(}\hlnum{10}\hlstd{,}\hlnum{5}\hlstd{)}\hlopt{*}\hlnum{5}\hlopt{^}\hlnum{5}\hlopt{/}\hlnum{6}\hlopt{^}\hlnum{10}
\end{alltt}
\begin{verbatim}
## [1] 0.01302381
\end{verbatim}
\end{kframe}
\end{knitrout}

Ahora $X$ número de veces que hay que tirar los 10 dados hasta obtenr una tirada con exactamente 5 cincos ("éxito")  claramente es la repetición de un experimento $Ber(p=0.0130238)$  luego $X$ sigue una ley $Ge(p=P(Y=5)=0.0130238).$



\probl La probabilidad de que un cierto anunció de una página web reciba un \textsl{clic}  de un usuario y lo vea  es de $p=0.75$ por cada acceso a la página web. Su pongamos que   20 personas, de forma independiente,  visitan esa página con ese anuncio,
contestar a las siguientes preguntas (\textbf{\textsc{UTILIZAD EL CÓDIGO DE LA PÁGINA SIGUIENTE}}):
\begin{enumerate}[a)]
\item Sea $X$ la variable aleatoria que cuenta el número de  clientes que no  visitan el anuncio  e $Y$ la de clientes que sí visitan el anuncio ¿Cuáles son las distribuciones de $X$ y  de $Y$? (\textbf{1.25 punto}).
\item ¿Cuál es la probabilidad de que ningún cliente vea el anuncio?.(\textbf{1.25 punto}).
\item ¿Cuál es la probabilidad de que vean el anuncio  más de 2 clientes  y menos de 5?(\textbf{1 punto.})
\item ¿Cuál es el número esperado de visualizaciones?(\textbf{1 punto}).
\end{enumerate}

\textbf{Solución:}

Bajo estas condiciones 

La variable $X$= número de clientes que NO visitan el anuncio es una $B(n=20,p_{\mbox{no}}=1-p=0.25)$ y la del número de clientes que SÍ lo visitan $Y$ sigue una distribución $B(n=20,p_{\mbox{sí}}=p=0.75)$.




$$P(\mbox{ningún cliente entre los 20 vea el anuncio})=P(X=0)=0.0031712.$$

lo hemos calculado con R

\begin{knitrout}
\definecolor{shadecolor}{rgb}{0.969, 0.969, 0.969}\color{fgcolor}\begin{kframe}
\begin{alltt}
\hlkwd{dbinom}\hlstd{(}\hlnum{0}\hlstd{,}\hlkwc{size}\hlstd{=}\hlnum{20}\hlstd{,}\hlkwc{prob}\hlstd{=}\hlnum{1}\hlopt{-}\hlnum{0.75}\hlstd{)}
\end{alltt}
\begin{verbatim}
## [1] 0.003171212
\end{verbatim}
\end{kframe}
\end{knitrout}

$$P(\mbox{vean el anuncio  más de 2 clientes  y menos de 5})=P(Y=3)+P(Y=4),$$

utilizando el código adjunto de la hoja 2 del examen
\begin{knitrout}
\definecolor{shadecolor}{rgb}{0.969, 0.969, 0.969}\color{fgcolor}\begin{kframe}
\begin{alltt}
\hlkwd{dbinom}\hlstd{(}\hlnum{0}\hlopt{:}\hlnum{4}\hlstd{,}\hlkwc{size}\hlstd{=}\hlnum{20}\hlstd{,}\hlkwc{p}\hlstd{=}\hlnum{0.75}\hlstd{)}
\end{alltt}
\begin{verbatim}
## [1] 9.094947e-13 5.456968e-11 1.555236e-09 2.799425e-08 3.569266e-07
\end{verbatim}
\end{kframe}
\end{knitrout}

tenemos que 


$$
P(Y=3)+P(Y=4)=2.799425\times 10^{-8}+ 3.569266\times 10^{-7}=
\ensuremath{3.8492085\times 10^{-7}}.
$$


También se podía calcular  de esta otra forma

\begin{eqnarray*}
P(\mbox{vean el anuncio más de 2 clientes y menos de 5})&=& P(2<Y<5)
\\ &=&
P(2<Y\leq 4)= P(Y\leq 4)-P(Y\leq 2)\\
&=& \ensuremath{3.8653161\times 10^{-7}}- \ensuremath{1.6107151\times 10^{-9}}
\\ 
&=&
\ensuremath{3.849209\times 10^{-7}},
\end{eqnarray*}

con el código  siguiente:

\begin{knitrout}
\definecolor{shadecolor}{rgb}{0.969, 0.969, 0.969}\color{fgcolor}\begin{kframe}
\begin{alltt}
\hlkwd{pbinom}\hlstd{(}\hlnum{2}\hlstd{,}\hlkwc{size}\hlstd{=}\hlnum{20}\hlstd{,}\hlkwc{prob}\hlstd{=}\hlnum{0.75}\hlstd{)}
\end{alltt}
\begin{verbatim}
## [1] 1.610715e-09
\end{verbatim}
\begin{alltt}
\hlkwd{pbinom}\hlstd{(}\hlnum{4}\hlstd{,}\hlkwc{size}\hlstd{=}\hlnum{20}\hlstd{,}\hlkwc{prob}\hlstd{=}\hlnum{0.75}\hlstd{)}
\end{alltt}
\begin{verbatim}
## [1] 3.865316e-07
\end{verbatim}
\begin{alltt}
\hlkwd{pbinom}\hlstd{(}\hlnum{4}\hlstd{,}\hlkwc{size}\hlstd{=}\hlnum{20}\hlstd{,}\hlkwc{prob}\hlstd{=}\hlnum{0.75}\hlstd{)}\hlopt{-}\hlkwd{pbinom}\hlstd{(}\hlnum{2}\hlstd{,}\hlkwc{size}\hlstd{=}\hlnum{20}\hlstd{,}\hlkwc{prob}\hlstd{=}\hlnum{0.75}\hlstd{)}
\end{alltt}
\begin{verbatim}
## [1] 3.849209e-07
\end{verbatim}
\end{kframe}
\end{knitrout}

El número esperado de visualizaciones es 
$$E(Y)=20\cdot 0.75=15.$$

\probl Una variable aleatoria sigue una ley   de distribución  en el intervalo $(0,1]$ si función de densidad es, para algún número real $\alpha>0$:

$$
f_X(x)=\left\{
\begin{array}{ll}
\alpha \cdot (1-x)  & \mbox{ si } 0 < x < 1 \\
0 & \mbox{ en otro caso.}
\end{array}
\right.
$$

\begin{enumerate}[a)]
\item Calcular $\alpha$ para que $X$ sea densidad (\textbf{1.25 punto.})
\item Calcular su función de distribución (\textbf{1.25 punto.}).
\item Calcular $E(X)$ y $E\left(\frac{X-1}{2}\right)$(\textbf{1 punto.}).
\item Calcular el cuantil $x_{0.5}$ (\textbf{1 punto}).
\end{enumerate}

\textbf{Solución:}



\begin{eqnarray*}
1 &=&\displaystyle \int_{-\infty}^{+\infty} f_X(x) dx=\int_{0}^1 \alpha\cdot (1-x) dx\\
&=& \left[\alpha\cdot \left(x-\frac{x^2}{2}\right)\right]_{x=0}^{x=1}= \alpha\cdot \left(\left(1-\frac{1^2}{2}\right)-\left(0-\frac{0^2}{2}\right)\right)=\frac{\alpha}{2}.
\end{eqnarray*}

Luego $\frac{\alpha}{2}=1$ y entonces $\alpha=2.$ La función de densidad es:

$$
f_X(x)=\left\{
\begin{array}{ll}
2 \cdot (1-x)  & \mbox{ si } 0 < x < 1 \\
0 & \mbox{ en otro caso.}
\end{array}
\right.
$$


La función de distribución es $F_X(x)=\inf_{-\infty}^c f_X(t) dt$ tenemos tres casos:



$$
F_X(x)=P(X\leq x)=\left\{
\begin{array}{ll}
0 & \mbox{, si } x\leq 0 \\
\int_{0}^{x} 2\cdot(1-t) dt =\left[2\cdot \left(t-\frac{t^2}{2}\right)\right]_{t=0}^{t=x}
= 2\cdot \left(x-\frac{x^2}{2}\right) & \mbox{, si } 0<x<1 \\
1 & \mbox{, si } x\geq 1 .\\
\end{array}
\right.
$$


\begin{eqnarray*}
E(X)&=&\int_{-\infty}^{+\infty} x\cdot f_X(x) dx=\int_{0}^{1} x\cdot 2\cdot (1-x) dx=
\int_{0}^{1} 2\cdot (x-x^2) dx\\
&=& \left[2\cdot \left(\frac{x^2}{2}-\frac{x^3}{3}\right) \right]_{x=0}^{x=1}= 2\cdot \left(\frac{1^2}{2}-\frac{1^3}{3}\right)=2\cdot \left(\frac{3-2}{6}\right)=\frac13.
\end{eqnarray*}


Nos piden el cuantil $0.5$ $x_0.5$ es el valor tal que 

$$0.5=F_X(x_{0.5})=2\cdot \left(x_{0.5}-\frac{_{0.5}^2}{2}\right)$$

Operando obtenemos la ecuación de segundo grado 

$$x_{0.5}^2-2\cdot x_{0.5}+0.5=0,$$

sus soluciones son 

$$
x_{0.5}=\frac{-(-2) \pm\sqrt{(-2)^2 -4\cdot 1\cdot 0.5}}{2\cdot 1}=
\frac{2 \pm\sqrt{2}}{2}=
\left\{\begin{array}{l}
1.7071068\\
0.2928932
\end{array} \right.,
$$


luego $x_{0.5}=0.2928932$ pues la otra solución es mayor que $1$ y  no está en el dominio de $X$.





\newpage

\textbf{Código problema 2}



\begin{knitrout}\scriptsize
\definecolor{shadecolor}{rgb}{0.969, 0.969, 0.969}\color{fgcolor}\begin{kframe}
\begin{alltt}
\hlkwd{choose}\hlstd{(}\hlnum{100}\hlstd{,}\hlnum{5}\hlstd{)}\hlopt{*}\hlnum{5}\hlopt{^}\hlnum{5}
\end{alltt}
\begin{verbatim}
## [1] 235273500000
\end{verbatim}
\begin{alltt}
\hlnum{6}\hlopt{^}\hlnum{10}
\end{alltt}
\begin{verbatim}
## [1] 60466176
\end{verbatim}
\begin{alltt}
\hlkwd{dbinom}\hlstd{(}\hlnum{5}\hlstd{,}\hlkwc{size}\hlstd{=}\hlnum{10}\hlstd{,}\hlkwc{prob}\hlstd{=}\hlnum{1}\hlopt{/}\hlnum{6}\hlstd{)}
\end{alltt}
\begin{verbatim}
## [1] 0.01302381
\end{verbatim}
\end{kframe}
\end{knitrout}


\textbf{Código problema 3:}

\begin{knitrout}\scriptsize
\definecolor{shadecolor}{rgb}{0.969, 0.969, 0.969}\color{fgcolor}\begin{kframe}
\begin{alltt}
\hlkwd{dbinom}\hlstd{(}\hlnum{0}\hlopt{:}\hlnum{4}\hlstd{,}\hlkwc{size}\hlstd{=}\hlnum{20}\hlstd{,}\hlkwc{prob}\hlstd{=}\hlnum{0.75}\hlstd{)}
\end{alltt}
\begin{verbatim}
## [1] 9.094947e-13 5.456968e-11 1.555236e-09 2.799425e-08 3.569266e-07
\end{verbatim}
\begin{alltt}
\hlnum{1}\hlopt{-}\hlkwd{dbinom}\hlstd{(}\hlnum{0}\hlopt{:}\hlnum{4}\hlstd{,}\hlkwc{size}\hlstd{=}\hlnum{20}\hlstd{,}\hlkwc{prob}\hlstd{=}\hlnum{0.75}\hlstd{)}
\end{alltt}
\begin{verbatim}
## [1] 1.0000000 1.0000000 1.0000000 1.0000000 0.9999996
\end{verbatim}
\begin{alltt}
\hlkwd{dbinom}\hlstd{(}\hlnum{0}\hlopt{:}\hlnum{4}\hlstd{,}\hlkwc{size}\hlstd{=}\hlnum{20}\hlstd{,}\hlkwc{prob}\hlstd{=}\hlnum{1}\hlopt{-}\hlnum{0.75}\hlstd{)}
\end{alltt}
\begin{verbatim}
## [1] 0.003171212 0.021141413 0.066947808 0.133895615 0.189685455
\end{verbatim}
\begin{alltt}
\hlnum{1}\hlopt{-}\hlkwd{dbinom}\hlstd{(}\hlnum{0}\hlopt{:}\hlnum{4}\hlstd{,}\hlkwc{size}\hlstd{=}\hlnum{20}\hlstd{,}\hlkwc{prob}\hlstd{=}\hlnum{1}\hlopt{-}\hlnum{0.75}\hlstd{)}
\end{alltt}
\begin{verbatim}
## [1] 0.9968288 0.9788586 0.9330522 0.8661044 0.8103145
\end{verbatim}
\begin{alltt}
\hlkwd{pbinom}\hlstd{(}\hlnum{1}\hlstd{,}\hlkwc{size}\hlstd{=}\hlnum{20}\hlstd{,}\hlkwc{prob}\hlstd{=}\hlnum{0.75}\hlstd{,}\hlkwc{lower.tail} \hlstd{=} \hlnum{FALSE}\hlstd{)}
\end{alltt}
\begin{verbatim}
## [1] 1
\end{verbatim}
\begin{alltt}
\hlkwd{pbinom}\hlstd{(}\hlnum{3}\hlstd{,}\hlkwc{size}\hlstd{=}\hlnum{20}\hlstd{,}\hlkwc{prob}\hlstd{=}\hlnum{0.75}\hlstd{,}\hlkwc{lower.tail} \hlstd{=} \hlnum{FALSE}\hlstd{)}
\end{alltt}
\begin{verbatim}
## [1] 1
\end{verbatim}
\begin{alltt}
\hlkwd{pbinom}\hlstd{(}\hlnum{4}\hlstd{,}\hlkwc{size}\hlstd{=}\hlnum{20}\hlstd{,}\hlkwc{prob}\hlstd{=}\hlnum{0.75}\hlstd{,}\hlkwc{lower.tail} \hlstd{=} \hlnum{TRUE}\hlstd{)}
\end{alltt}
\begin{verbatim}
## [1] 3.865316e-07
\end{verbatim}
\begin{alltt}
\hlkwd{pbinom}\hlstd{(}\hlnum{5}\hlstd{,}\hlkwc{size}\hlstd{=}\hlnum{20}\hlstd{,}\hlkwc{prob}\hlstd{=}\hlnum{0.75}\hlstd{,}\hlkwc{lower.tail} \hlstd{=} \hlnum{TRUE}\hlstd{)}
\end{alltt}
\begin{verbatim}
## [1] 3.813027e-06
\end{verbatim}
\begin{alltt}
\hlkwd{pbinom}\hlstd{(}\hlnum{4}\hlstd{,}\hlkwc{size}\hlstd{=}\hlnum{20}\hlstd{,}\hlkwc{prob}\hlstd{=}\hlnum{0.75}\hlstd{)}
\end{alltt}
\begin{verbatim}
## [1] 3.865316e-07
\end{verbatim}
\begin{alltt}
\hlkwd{pbinom}\hlstd{(}\hlnum{5}\hlstd{,}\hlkwc{size}\hlstd{=}\hlnum{20}\hlstd{,}\hlkwc{prob}\hlstd{=}\hlnum{0.75}\hlstd{)}
\end{alltt}
\begin{verbatim}
## [1] 3.813027e-06
\end{verbatim}
\end{kframe}
\end{knitrout}
\end{document}

