\documentclass[12pt]{article}\usepackage[]{graphicx}\usepackage[]{color}
% maxwidth is the original width if it is less than linewidth
% otherwise use linewidth (to make sure the graphics do not exceed the margin)
\makeatletter
\def\maxwidth{ %
  \ifdim\Gin@nat@width>\linewidth
    \linewidth
  \else
    \Gin@nat@width
  \fi
}
\makeatother

\definecolor{fgcolor}{rgb}{0.345, 0.345, 0.345}
\newcommand{\hlnum}[1]{\textcolor[rgb]{0.686,0.059,0.569}{#1}}%
\newcommand{\hlstr}[1]{\textcolor[rgb]{0.192,0.494,0.8}{#1}}%
\newcommand{\hlcom}[1]{\textcolor[rgb]{0.678,0.584,0.686}{\textit{#1}}}%
\newcommand{\hlopt}[1]{\textcolor[rgb]{0,0,0}{#1}}%
\newcommand{\hlstd}[1]{\textcolor[rgb]{0.345,0.345,0.345}{#1}}%
\newcommand{\hlkwa}[1]{\textcolor[rgb]{0.161,0.373,0.58}{\textbf{#1}}}%
\newcommand{\hlkwb}[1]{\textcolor[rgb]{0.69,0.353,0.396}{#1}}%
\newcommand{\hlkwc}[1]{\textcolor[rgb]{0.333,0.667,0.333}{#1}}%
\newcommand{\hlkwd}[1]{\textcolor[rgb]{0.737,0.353,0.396}{\textbf{#1}}}%
\let\hlipl\hlkwb

\usepackage{framed}
\makeatletter
\newenvironment{kframe}{%
 \def\at@end@of@kframe{}%
 \ifinner\ifhmode%
  \def\at@end@of@kframe{\end{minipage}}%
  \begin{minipage}{\columnwidth}%
 \fi\fi%
 \def\FrameCommand##1{\hskip\@totalleftmargin \hskip-\fboxsep
 \colorbox{shadecolor}{##1}\hskip-\fboxsep
     % There is no \\@totalrightmargin, so:
     \hskip-\linewidth \hskip-\@totalleftmargin \hskip\columnwidth}%
 \MakeFramed {\advance\hsize-\width
   \@totalleftmargin\z@ \linewidth\hsize
   \@setminipage}}%
 {\par\unskip\endMakeFramed%
 \at@end@of@kframe}
\makeatother

\definecolor{shadecolor}{rgb}{.97, .97, .97}
\definecolor{messagecolor}{rgb}{0, 0, 0}
\definecolor{warningcolor}{rgb}{1, 0, 1}
\definecolor{errorcolor}{rgb}{1, 0, 0}
\newenvironment{knitrout}{}{} % an empty environment to be redefined in TeX

\usepackage{alltt}
\usepackage{amsfonts,amssymb,amsmath,amsthm,graphicx,enumerate}
\usepackage[utf8]{inputenc}
\usepackage[T1]{fontenc}        
\usepackage[spanish]{babel}
\decimalpoint
\advance\hoffset by -0.9in
\advance\textwidth by 1.8in
\advance\voffset by -1in
\advance\textheight by 2in
\parskip= 1 ex
\parindent = 10pt
\baselineskip= 13pt
\newcommand{\red}[1]{\textcolor{red}{#1}}


\renewcommand{\leq}{\leqslant}
\renewcommand{\geq}{\geqslant}

\newcounter{problemes}
\newcounter{punts} \def\thepunts{\arabic{punts}}
\def\probl{\addtocounter{problemes}{1} \setcounter{punts}{0}
\medskip\noindent{\bf \theproblemes) }}
\def\punt{\addtocounter{punts}{1} \smallskip{\emph{\thepunts) }}}

\newcommand{\novapart}{\noindent\hrulefill}
\newcommand{\VV}{\textbf{\Large \checkmark}}
\newcommand{\coment}[1]{\noindent{\footnotesize\textbf{Comentario}: #1\par}}
\newcommand{\sol}[1]{{\footnotesize #1\par}}

\renewcommand{\VV}{}
\renewcommand{\sol}[1]{}
\renewcommand{\coment}[1]{}


\pagestyle{empty}
\IfFileExists{upquote.sty}{\usepackage{upquote}}{}
\begin{document}
%\SweaveOpts{concordance=TRUE}
%1
\noindent\emph{Nombre:}\hfill\hfill\hfill\hfill\hfill\hfill\hfill\ \emph{Grupo:}\hfill \vspace*{-2ex}

\begin{center}
\textsc{Matemáticas III. GIN2. Control Parcial abril  2020-2021.}
\end{center}

\setcounter{problemes}{0}

\probl  Consideremos los siguientes sucesos $A$, $B$ tales que
$P(A|B)=0.4$, $P(A|B^c)=0.7$, $P(B^c)=0.2$. Calcular $P(A)$ y $P(B|A)$.
(\textbf{1 punto}).


\probl \textbf{\textsc{PUNTO EXTRA EN ESTE EXAMEN}}. Tiramos 10 dados de parchís hasta obtener exactamente 5 cincos incluido ese último lanzamiento.
Sea $X$  el número de tiradas necesarias ¿Cuál es la distribución de $X$ su valor esperado y su varianza? (\textbf{1 punto}).

\probl La probabilidad de que un cierto anunció de una página web reciba un \textsl{clic}  de un usuario y lo vea  es de $p=0.75$ por cada acceso a la página web. Su pongamos que   20 personas, de forma independiente,  visitan esa página con ese anuncio,
contestar a las siguientes preguntas (\textbf{\textsc{UTILIZAD EL CÓDIGO DE LA PÁGINA SIGUIENTE}}):
\begin{enumerate}[a)]
\item Sea $X$ la variable aleatoria que cuenta el número de  clientes que no  visitan el anuncio  e $Y$ la de clientes que sí visitan el anuncio ¿Cuáles son las distribuciones de $X$ y  de $Y$? (\textbf{1.25 punto}).
\item ¿Cuál es la probabilidad de que ningún cliente vea el anuncio?.(\textbf{1.25 punto}).
\item ¿Cuál es la probabilidad de que vean el anuncio  más de 2 clientes  y menos de 5?(\textbf{1 punto.})
\item ¿Cuál es el número esperado de visualizaciones?(\textbf{1 punto}).
\end{enumerate}


\probl Una variable aleatoria sigue una ley   de distribución  en el intervalo $(0,1]$ si función de densidad es, para algún número real $\alpha>0$:

$$
f(x)=\left\{
\begin{array}{ll}
\alpha \cdot (1-x)  & \mbox{ si } 0 < x < 1 \\
0 & \mbox{ en otro caso.}
\end{array}
\right.
$$

\begin{enumerate}[a)]
\item Calcular $\alpha$ para que $X$ sea densidad (\textbf{1.25 punto.})
\item Calcular su función de distribución (\textbf{1.25 punto.}).
\item Calcular $E(X)$ y $E\left(\frac{X-1}{2}\right)$(\textbf{1 punto.}).
\item Calcular el cuantil $x_{0.5}$ (\textbf{1 punto}).
\end{enumerate}


\newpage

\textbf{Código problema 2}



\begin{knitrout}\scriptsize
\definecolor{shadecolor}{rgb}{0.969, 0.969, 0.969}\color{fgcolor}\begin{kframe}
\begin{alltt}
\hlkwd{choose}\hlstd{(}\hlnum{100}\hlstd{,}\hlnum{5}\hlstd{)}\hlopt{*}\hlnum{5}\hlopt{^}\hlnum{5}
\end{alltt}
\begin{verbatim}
## [1] 235273500000
\end{verbatim}
\begin{alltt}
\hlnum{6}\hlopt{^}\hlnum{10}
\end{alltt}
\begin{verbatim}
## [1] 60466176
\end{verbatim}
\begin{alltt}
\hlkwd{dbinom}\hlstd{(}\hlnum{5}\hlstd{,}\hlkwc{size}\hlstd{=}\hlnum{10}\hlstd{,}\hlkwc{prob}\hlstd{=}\hlnum{1}\hlopt{/}\hlnum{6}\hlstd{)}
\end{alltt}
\begin{verbatim}
## [1] 0.01302381
\end{verbatim}
\end{kframe}
\end{knitrout}


\textbf{Código problema 3:}

\begin{knitrout}\scriptsize
\definecolor{shadecolor}{rgb}{0.969, 0.969, 0.969}\color{fgcolor}\begin{kframe}
\begin{alltt}
\hlkwd{dbinom}\hlstd{(}\hlnum{0}\hlopt{:}\hlnum{4}\hlstd{,}\hlkwc{size}\hlstd{=}\hlnum{20}\hlstd{,}\hlkwc{prob}\hlstd{=}\hlnum{0.75}\hlstd{)}
\end{alltt}
\begin{verbatim}
## [1] 9.094947e-13 5.456968e-11 1.555236e-09 2.799425e-08 3.569266e-07
\end{verbatim}
\begin{alltt}
\hlnum{1}\hlopt{-}\hlkwd{dbinom}\hlstd{(}\hlnum{0}\hlopt{:}\hlnum{4}\hlstd{,}\hlkwc{size}\hlstd{=}\hlnum{20}\hlstd{,}\hlkwc{prob}\hlstd{=}\hlnum{0.75}\hlstd{)}
\end{alltt}
\begin{verbatim}
## [1] 1.0000000 1.0000000 1.0000000 1.0000000 0.9999996
\end{verbatim}
\begin{alltt}
\hlkwd{dbinom}\hlstd{(}\hlnum{0}\hlopt{:}\hlnum{4}\hlstd{,}\hlkwc{size}\hlstd{=}\hlnum{20}\hlstd{,}\hlkwc{prob}\hlstd{=}\hlnum{1}\hlopt{-}\hlnum{0.75}\hlstd{)}
\end{alltt}
\begin{verbatim}
## [1] 0.003171212 0.021141413 0.066947808 0.133895615 0.189685455
\end{verbatim}
\begin{alltt}
\hlnum{1}\hlopt{-}\hlkwd{dbinom}\hlstd{(}\hlnum{0}\hlopt{:}\hlnum{4}\hlstd{,}\hlkwc{size}\hlstd{=}\hlnum{20}\hlstd{,}\hlkwc{prob}\hlstd{=}\hlnum{1}\hlopt{-}\hlnum{0.75}\hlstd{)}
\end{alltt}
\begin{verbatim}
## [1] 0.9968288 0.9788586 0.9330522 0.8661044 0.8103145
\end{verbatim}
\begin{alltt}
\hlkwd{pbinom}\hlstd{(}\hlnum{1}\hlstd{,}\hlkwc{size}\hlstd{=}\hlnum{20}\hlstd{,}\hlkwc{prob}\hlstd{=}\hlnum{0.75}\hlstd{,}\hlkwc{lower.tail} \hlstd{=} \hlnum{FALSE}\hlstd{)}
\end{alltt}
\begin{verbatim}
## [1] 1
\end{verbatim}
\begin{alltt}
\hlkwd{pbinom}\hlstd{(}\hlnum{3}\hlstd{,}\hlkwc{size}\hlstd{=}\hlnum{20}\hlstd{,}\hlkwc{prob}\hlstd{=}\hlnum{0.75}\hlstd{,}\hlkwc{lower.tail} \hlstd{=} \hlnum{FALSE}\hlstd{)}
\end{alltt}
\begin{verbatim}
## [1] 1
\end{verbatim}
\begin{alltt}
\hlkwd{pbinom}\hlstd{(}\hlnum{4}\hlstd{,}\hlkwc{size}\hlstd{=}\hlnum{20}\hlstd{,}\hlkwc{prob}\hlstd{=}\hlnum{0.75}\hlstd{,}\hlkwc{lower.tail} \hlstd{=} \hlnum{TRUE}\hlstd{)}
\end{alltt}
\begin{verbatim}
## [1] 3.865316e-07
\end{verbatim}
\begin{alltt}
\hlkwd{pbinom}\hlstd{(}\hlnum{5}\hlstd{,}\hlkwc{size}\hlstd{=}\hlnum{20}\hlstd{,}\hlkwc{prob}\hlstd{=}\hlnum{0.75}\hlstd{,}\hlkwc{lower.tail} \hlstd{=} \hlnum{TRUE}\hlstd{)}
\end{alltt}
\begin{verbatim}
## [1] 3.813027e-06
\end{verbatim}
\begin{alltt}
\hlkwd{pbinom}\hlstd{(}\hlnum{4}\hlstd{,}\hlkwc{size}\hlstd{=}\hlnum{20}\hlstd{,}\hlkwc{prob}\hlstd{=}\hlnum{0.75}\hlstd{)}
\end{alltt}
\begin{verbatim}
## [1] 3.865316e-07
\end{verbatim}
\begin{alltt}
\hlkwd{pbinom}\hlstd{(}\hlnum{5}\hlstd{,}\hlkwc{size}\hlstd{=}\hlnum{20}\hlstd{,}\hlkwc{prob}\hlstd{=}\hlnum{0.75}\hlstd{)}
\end{alltt}
\begin{verbatim}
## [1] 3.813027e-06
\end{verbatim}
\end{kframe}
\end{knitrout}






\end{document}

